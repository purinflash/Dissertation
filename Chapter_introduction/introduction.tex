\chapter{Introduction} \label{ch:introduction}

\ifpdf
    \graphicspath{{Chapter_introduction/figures/PNG/}{Chapter_introduction/figures/PDF/}{Chapter_introduction/figures/}}
\else
    \graphicspath{{Chapter_introduction/figures/EPS/}{Chapter_introduction/figures/}}
\fi

Upper tropospheric ozone has significant impacts on the radiative and chemical budgets of the atmosphere \citep{Kiehl:1999uq}. Global tropospheric ozone burden has seen an increase of 71--130\,\unit{Tg} since preindustrial period, with much of the uncertainties coming from the estimation of preindustrial emission scenarios for anthropogenic, biomass burning, and lightning sources \citep[][and references therein]{Lamarque:2005gb}. The radiative forcing resulting from this increase depends strongly on the vertical distribution and is the most sensitive near the tropopause \citep{Lacis:1990fk}. \citet{Gauss:2006zr} calculated that the global net radiative forcing resulting from a 7.9--13.8\,\unit{DU} increase in tropospheric ozone (ignoring stratospheric changes) varies between 0.25 and 0.45\,\unit{W\,m^{-2}}. In particular, several models used in the same study as well as \citet{Stevenson:1998fk} estimated up to 0.5--1.1\,\unit{W \,m^{-2}} increase over parts of northern mid-latitudes such as North America and the Mediterraneans. The ozone increases over these regions are also verified by ozonesonde records spanning periods between 1970's and 2004 \citep{Oltmans:2006kc}.

% Lamarque:2005 - trop ozone evolution since pre-industrial. Summary of other studies. 77 -- 130 Tg
% Stevenson:1998 - radiative forcing due to trop ozone increase. max 30 ppbv
% Lacis:1990 - vertical distribution. Most sensitive at tropopause
% Gauss:2006 - multi-model study showing 0.25--0.45 W/m2 increase in global net radiative forcing
% Oltmans:2006 - global ozonesonde study verifying northern mid-lat ozone increase

Recent studies have identified monsoonal ozone precursor accumulations above North America \citep[][and references therein]{Li:2005ss,Cooper:2009nx}, Asia \citep{Park:2007bh,Worden:2009ve}, and equatorial Africa \citep{Bouarar:2011ly} during summertime in the upper troposphere. \citet{Cooper:2007cr} calculated a median tropospheric ozone mixing ratio of 87~\unit{ppbv} (after removing stratospheric intrusion) above Huntsville,~AL in August 2006. This is about twice the value measured along the United States west coast during the same period. The observed upper tropospheric ozone enhancement is linked to the North American Monsoon anticyclonic circulation, which traps ozone precursors that subsequently enhances ozone production \citep{Li:2005ss}. \citet{Cooper:2006dq} estimated that up to 84\% of the enhancement observed above southern United States can be attributed to {\insitu} ozone production from lightning-produced \chem{NO_x} (L\chem{NO_x}). The radiative forcing of this enhancement is +0.50\,\unit{W\,m^{-2}}, 70\% or which attributable to enhancement through L\chem{NO_x}.

By leveraging improved lightning parameterization, online budgeting diagnostics, and passive tracers, this thesis intends to extend the result from previous studies and understand various contributing factors that led to the observed ozone distribution. The primary goals of this study are as follow:

\begin{enumerate}
\item{} Simulate the observed ozone distribution using a regional chemistry transport model and evaluate the outputs against independent observations;
\item{} Utilize online budgeting diagnostics and passive tracers to understand the evolution of the simulated ozone distribution;
\item{} Conduct sensitivity simulations to evaluate how the ozone enhancement would react to different emission scenarios from anthropogenic, biogenic, and lightning sources;
\item{} Evaluate outputs from a multiyear simulation to understand the inter-annual variability of the ozone enhancement and potential impacts from future emission scenarios.
\end{enumerate}

\newpage
\section{Tropospheric ozone chemistry} \label{sec:intro/ozone}

Since the primary goals of this study involve diagnosing the simulated ozone values, it is crucial to understand the basis and complexity of tropospheric ozone chemistry. Through its production-destruction cycle, ozone interacts with dozens of chemical species in the troposphere. In the presence of water and shortwave radiation, photodissociation of ozone is also a precursor for hydroxl (\chem{OH}) radicals, which is a primary oxidizing agent for many volatile organic compounds (VOCs) in the atmosphere:
\begin{eqnarray}
	\chem{O_3} + h\nu(\lambda<330\,\unit{nm}) &\rightarrow& \chem{O(^1D)} + \chem{O_2} \label{rxn:j(o3)} \\
	\chem{O(^1D)} + \chem{H_2O} &\rightarrow& 2\chem{OH} \label{rxn:o1d+h2o}
\end{eqnarray}
Therefore, ozone is the primary oxidant in the tropospheric by either directly or indirectly (through \chem{OH}) oxidizing many trace gases, many of which exclusively originate from anthropogenic and biogenic emissions. Consequently, the lifetime of these gases are also dependent on the ozone concentration.

In a low-hydrocarbon atmosphere, the following reactions determine the steady-state level of ozone:
\begin{eqnarray}
	\chem{NO_2} + h\nu &\rightarrow& \chem{NO} + \chem{O(^3P}) \label{rxn:j(no2)} \\
	\chem{O(^3P)} + \chem{O_2} + \chem{M} &\rightarrow& \chem{O_3} + \chem{M} \label{rxn:o3p+o2} \\
	\chem{O_3} + \chem{NO} &\rightarrow& \chem{NO_2} + \chem{O_2} \label{rxn:o3+no}
\end{eqnarray}
Reaction \ref{rxn:o3p+o2} is a termolecular reaction consuming a ground-state oxygen atom produced primarily from photolysis of \chem{NO_2} (Reaction \ref{rxn:j(no2)}). However, the produced \chem{O_3}, or any other ozone molecule in the environment, may react with \chem{NO} to form \chem{NO_2} again (Reaction \ref{rxn:o3+no}). This set of reactions does not destroy or produce ozone molecules, thus it implies a steady-state of ozone, and subsequently \chem{OH}, that is governed by the photolysis rate of \chem{NO_2} and may be characterized by $[\chem{O_3}] \approx (J_{\ref{rxn:j(no2)}}/k_{\ref{rxn:o3+no}})([\chem{NO_2}]/[\chem{NO}])$, where $J_{\ref{rxn:j(no2)}}$ is the photolysis rate constant for Reaction \ref{rxn:j(no2)} and $k_{\ref{rxn:o3+no}}$ is the rate constant for Reaction \ref{rxn:o3+no}.

In the presence of volatile organic compounds (VOCs) and hydroxy radicals (\chem{HO}), however, the reaction set becomes much more complicated. VOCs may be oxidized to produce hydroperoxy radicals (\chem{HO_2}), e.g. from \chem{CO} and \chem{HCHO}, or other peroxy radicals with higher carbon numbers (\chem{RO_2}), e.g. methyl peroxy radical (\chem{CH_3O_2}). These products may then compete with Reaction \ref{rxn:o3+no} to consume the available \chem{NO} without the cost of an ozone molecule in the process:
\begin{eqnarray}
	\chem{HO_2} + \chem{NO} &\rightarrow& \chem{HO} + \chem{NO_2} \label{rxn:ho2+no} \\
	\chem{RO_2} + \chem{NO} &\rightarrow& \chem{RO} + \chem{NO_2} \label{rxn:ro2+no}
\end{eqnarray}
Of course, the produced \chem{NO_2} may now proceed to be photolyzed through Reaction \ref{rxn:j(no2)} and perpetuate the ozone cycle. Incidentally, an excess presence of \chem{HO_x} from the oxidation of VOCs is capable of reducing ozone level through two reactions:
\begin{eqnarray}
	\chem{O_3} + \chem{HO} &\rightarrow& \chem{HO_2} + \chem{O_2} \label{rxn:o3+ho} \\
	\chem{O_3} + \chem{HO_2} &\rightarrow& \chem{HO} + 2\chem{O_2} \label{rxn:o3+ho2}
\end{eqnarray}
Thus excessive presence of hydrocarbons without sufficient \chem{NO} to carry out Reactions \ref{rxn:ho2+no} and \ref{rxn:ro2+no} also has the potential of destroying ozone by shortcutting the ozone recycling reaction chain.

All the reactions between \chem{O_3}, \chem{NO_x}, \chem{HO_x} and VOCs result in an ozone level determined by the balance between \chem{NO_x} and VOCs concentrations as well as photolysis rates. Too much \chem{NO_x} without sufficient VOCs may cause ``\chem{NO_x}-titration,'' a process which reduces ozone concentration through Reaction \ref{rxn:o3+no}. On the other hand, in an environment with sufficient VOCs but \chem{NO_x} is scarce, ozone is also destroyed. Thus, for high ozone level to occur (such as in photochemical smogs), compatible levels of both \chem{NO_x} and VOCs are required. In theory, the variability of ozone concentration in a homogeneously mixed volume may be explained by looking at the ratio between \chem{NO_x} and VOCs, and identify the corresponding \chem{NO_x}-limited and VOC-limited regimes. In practice, transport and mixing complicates such approximation.


\subsection{Sources for Volatile organic compounds (VOCs)} \label{ssec:intro/ozone/voc}

The definition of volatile organic compounds, or VOCs, varies. For consistency, we use the following definition from the United States Environmental Protection Agency (EPA):

\begin{quotation}
Volatile organic compounds (VOC) means any compound of carbon, excluding carbon monoxide, carbon dioxide, carbonic acid, metallic carbides or carbonates, and ammonium carbonate, which participates in atmospheric photochemical reactions \citep{EPA:2010ve}.
\end{quotation}

The extended definition also explicitly excludes specific chemicals for ``negligible photochemical reactivity,'' e.g. methane, ethane, CFCs, HCFCs, HFCs. Sources for VOCs can be partitioned largely into anthropogenic and biogenic emissions, with sporadic contributions from biomass burning and volcanic eruptions.

According to the EPA's 2008 National Emission Inventory (NEI), the leading source for anthropogenic VOC are industrial processes, emitting $6.9\times10^6$ tons and contributing to a third of the national anthropogenic VOC budget including wild fires (Fig. \ref{fig:epa-nei-voc}). VOC emission from transportation was $5.6\times10^6$ tons (27\% of total). These numbers are to be compared to 2005 figures of $7.4\times10^6$ tons (36\% of total) for industrial processes and $6.9\times10^6$ tons (34\%) for transportation. Further reduction in emission from transportation is seen in 2011, down to $4.3\times10^6$ tons. However, emission from wild fires have drastically increased to $2.8\times10^6$ tons (4\%) in 2008 from $1.9\times10^6$ tons (3\%) in 2005. Also, emission from the ``miscellaneous'' category, which includes sources such as prescribed fire and gas stations, also increased from $3.3\times10^6$ tons (9\%) in 2005 to $4.7\times10^6$ tons (14\%) in 2008. Totals excluding wild fire were $18.3\times10^5$ tons for 2005 and $17.7\times10^6$ tons for 2008.

\figuremacroW{Fig01-01_NEI-VOC}{National Emission Inventory (NEI) VOC emission trend}{\label{fig:epa-nei-voc} Trend for total NEI VOC emission for 1970--2012. Data points prior to 1990 were reported every 5 years. 2006 and 2007 are interpolation of 2005 and 2008 data. After 2008, except for transportation, all other emissions were assumed constant. While some changes are real, some are due to changes in inventory methodology.}{1.0}

NEI2008 also estimated $31.7\times10^6$ tons of biogenic VOCs (BVOCs) in contiguous United States (CONUS). As anthropogenic VOC emission is being reduced through regulatory efforts, the impact of BVOCs on tropospheric chemistry is becoming more important. Of the hundreds of BVOCs identified, the global flux is dominated by isoprene (\chem{C_5H_8}) \citep{Guenther:2006kl}, which is primarily emitted by terrestrial foliage. Bottom-up global estimates of isoprene emission using Model for Emissions of Gases and Aerosols from Nature (MEGAN) based on leaf data gives about 600 Tg for 2003 \citep{Guenther:2006kl}, and its regional distribution has been further constrained with satellite observation of formaldehyde (\chem{HCHO}) column from satellite instruments such as the Global Ozone Monitoring Experiment (GOME) instrument on board of ERS-2 \citep{Palmer:2001nx, Palmer:2003cr,Palmer:2006qf},  the SCIAMACHY instrument on board of the ENVISAT  \citep{Dufour:2009fk, De-Smedt:2008uq, De-Smedt:2010kx}, and the Ozone Monitoring Instrument (OMI) on board of the EOS Aura satellite \citep{Millet:2008oq, Marais:2012kl}.


