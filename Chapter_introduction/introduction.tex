\chapter{Introduction} \label{ch:introduction}

\ifpdf
    \graphicspath{{Chapter_introduction/figures/PNG/}{Chapter_introduction/figures/PDF/}{Chapter_introduction/figures/}}
\else
    \graphicspath{{Chapter_introduction/figures/EPS/}{Chapter_introduction/figures/}}
\fi

\figuremacroN{Cooper-FTO3}{Filtered upper tropospheric ozone from \citet{Cooper:2007cr}}{\label{fig:cooper-fto3} Interpolated ozonesonde measurements between 10 and 11\,\unit{km} less stratospheric influence estimated using back trajectory \citep[after][Figure 2c]{Cooper:2007cr}.}

Upper tropospheric ozone has significant impacts on the radiative and chemical budgets of the atmosphere \citep{Kiehl:1999uq}. Global tropospheric ozone burden has seen an increase of 71--130\,\unit{Tg} since preindustrial period, with much of the uncertainties coming from the estimation of preindustrial emission scenarios for anthropogenic, biomass burning, and lightning sources \citep[][and references therein]{Lamarque:2005gb}. The radiative forcing resulting from this increase depends strongly on the vertical distribution and is the most sensitive near the tropopause \citep{Lacis:1990fk}. \citet{Gauss:2006zr} calculated that the global net radiative forcing resulting from a 7.9--13.8\,\unit{DU} increase in tropospheric ozone (ignoring stratospheric changes) varies between 0.25 and 0.45\,\unit{W\,m^{-2}}. In particular, several models cited by \citet{Gauss:2006zr} as well as \citet{Stevenson:1998fk} estimated up to 0.5--1.1\,\unit{W \,m^{-2}} increase over parts of northern mid-latitudes such as North America and the Mediterraneans. The ozone increases over these regions are also verified by ozonesonde records spanning periods between 1970's and 2004 \citep{Oltmans:2006kc}.

% Lamarque:2005 - trop ozone evolution since pre-industrial. Summary of other studies. 77 -- 130 Tg
% Stevenson:1998 - radiative forcing due to trop ozone increase. max 30 ppbv
% Lacis:1990 - vertical distribution. Most sensitive at tropopause
% Gauss:2006 - multi-model study showing 0.25--0.45 W/m2 increase in global net radiative forcing
% Oltmans:2006 - global ozonesonde study verifying northern mid-lat ozone increase

Recent studies have identified monsoonal ozone precursor accumulations above North America \citep[][and references therein]{Li:2005ss,Cooper:2009nx}, Asia \citep{Park:2007bh,Worden:2009ve}, and equatorial Africa \citep{Bouarar:2011ly} during summertime in the upper troposphere. \citet{Cooper:2007cr} calculated a median tropospheric ozone mixing ratio of 87~\unit{ppbv} (after removing stratospheric intrusion) above Huntsville,~AL in August 2006 (Figure \ref{fig:cooper-fto3}). This is about twice the value measured along the United States west coast during the same period. The observed upper tropospheric ozone enhancement is linked to the North American Monsoon anticyclonic circulation, which traps ozone precursors that subsequently enhances ozone production \citep{Li:2005ss}. \citet{Cooper:2006dq} estimated that up to 84\% of the enhancement observed above southern United States can be attributed to {\insitu} ozone production from lightning-produced \chem{NO_x} (L\chem{NO_x}). The radiative forcing of this enhancement is +0.50\,\unit{W\,m^{-2}}, 70\% or which attributable to enhancement through L\chem{NO_x}.

By leveraging improved lightning parameterization, online budgeting diagnostics, and passive tracers, this thesis intends to extend the result from previous studies and understand various contributing factors that led to the observed ozone distribution. The primary goals of this study are as follow:
\begin{enumerate}
\item{} Simulate the observed ozone distribution using a regional chemistry transport model and evaluate the outputs against independent observations;
\item{} Utilize online budgeting diagnostics and passive tracers to understand the evolution of the simulated ozone distribution;
\item{} Conduct sensitivity simulations to evaluate how the ozone enhancement would react to different emission scenarios from anthropogenic, biogenic, and lightning sources;
\item{} Evaluate outputs from a multiyear simulation to understand the inter-annual variability of the ozone enhancement and potential impacts from future emission scenarios.
\end{enumerate}

\newpage
\section{Tropospheric ozone chemistry} \label{sec:intro/ozone}

Since the primary goals of this study involve diagnosing the simulated ozone values, it is crucial to understand the basis and complexity of tropospheric ozone chemistry. Through its production-destruction cycle, ozone interacts with dozens of chemical species in the troposphere. In the presence of water and shortwave radiation, photodissociation of ozone is also a precursor for hydroxl (\chem{OH}) radicals, which is a primary oxidizing agent for many volatile organic compounds (VOCs) in the atmosphere:
\begin{eqnarray}
	\chem{O_3} + h\nu(\lambda<330\,\unit{nm}) &\rightarrow& \chem{O(^1D)} + \chem{O_2} \label{rxn:j(o3)} \\
	\chem{O(^1D)} + \chem{H_2O} &\rightarrow& 2\chem{OH} \label{rxn:o1d+h2o}
\end{eqnarray}
Therefore, ozone is the primary oxidant in the tropospheric by either directly or indirectly (through \chem{OH}) oxidizing trace gases, many of which exclusively originating from anthropogenic and biogenic emissions. Consequently, the lifetime of these gases are also dependent on the ozone concentration.

In a dry low-hydrocarbon atmosphere such as the stratosphere, the following reactions determine the steady-state level of ozone:
\begin{eqnarray}
	\chem{NO_2} + h\nu &\rightarrow& \chem{NO} + \chem{O(^3P}) \label{rxn:j(no2)} \\
	\chem{O(^3P)} + \chem{O_2} + \chem{M} &\rightarrow& \chem{O_3} + \chem{M} \label{rxn:o3p+o2} \\
	\chem{O_3} + \chem{NO} &\rightarrow& \chem{NO_2} + \chem{O_2} \label{rxn:o3+no}
\end{eqnarray}
Reaction \ref{rxn:o3p+o2} is a termolecular reaction consuming a ground-state oxygen atom produced primarily from photolysis of \chem{NO_2} (Reaction \ref{rxn:j(no2)}). However, the produced \chem{O_3}, or any other ozone molecule in the environment, may react with \chem{NO} to form \chem{NO_2} again (Reaction \ref{rxn:o3+no}). This set of reactions does not destroy or produce ozone molecules, thus it implies a steady-state of ozone, and subsequently \chem{OH}, that is governed by the photolysis rate of \chem{NO_2} and may be characterized by $[\chem{O_3}] \approx (J_{\ref{rxn:j(no2)}}/k_{\ref{rxn:o3+no}})([\chem{NO_2}]/[\chem{NO}])$, where $J_{\ref{rxn:j(no2)}}$ is the photolysis rate constant for Reaction \ref{rxn:j(no2)} and $k_{\ref{rxn:o3+no}}$ is the rate constant for Reaction \ref{rxn:o3+no}.

In the presence of volatile organic compounds (VOCs) and hydroxy radicals (\chem{HO}), however, the reaction set becomes much more complicated. VOCs may be oxidized to produce hydroperoxy radicals (\chem{HO_2}), e.g. from \chem{CO} and \chem{HCHO}, or other peroxy radicals with higher carbon numbers (\chem{RO_2}), e.g. methyl peroxy radical (\chem{CH_3O_2}). These products may then compete with Reaction \ref{rxn:o3+no} to consume the available \chem{NO} without the cost of an ozone molecule in the process:
\begin{eqnarray}
	\chem{HO_2} + \chem{NO} &\rightarrow& \chem{HO} + \chem{NO_2} \label{rxn:ho2+no} \\
	\chem{RO_2} + \chem{NO} &\rightarrow& \chem{RO} + \chem{NO_2} \label{rxn:ro2+no}
\end{eqnarray}
Of course, the produced \chem{NO_2} may now proceed to be photolyzed through Reaction \ref{rxn:j(no2)} and perpetuate the ozone cycle. Incidentally, an excess presence of \chem{HO_x} from the oxidation of VOCs is capable of reducing ozone level through two reactions:
\begin{eqnarray}
	\chem{O_3} + \chem{HO} &\rightarrow& \chem{HO_2} + \chem{O_2} \label{rxn:o3+ho} \\
	\chem{O_3} + \chem{HO_2} &\rightarrow& \chem{HO} + 2\chem{O_2} \label{rxn:o3+ho2}
\end{eqnarray}
Thus excessive presence of hydrocarbons without sufficient \chem{NO} to carry out Reactions \ref{rxn:ho2+no} and \ref{rxn:ro2+no} also has the potential of destroying ozone by shortcutting the ozone recycling reaction chain.

All the reactions between \chem{O_3}, \chem{NO_x}, \chem{HO_x} and VOCs result in an ozone level determined by the balance between \chem{NO_x} and VOCs concentrations as well as photolysis rates. Too much \chem{NO_x} without sufficient VOCs may cause ``\chem{NO_x}-titration,'' a process which reduces ozone concentration through Reaction \ref{rxn:o3+no}. On the other hand, in an environment with sufficient VOCs but \chem{NO_x} is scarce, ozone is also destroyed. Thus, for high ozone level to occur (such as in photochemical smogs), compatible levels of both \chem{NO_x} and VOCs are required. In theory, the variability of ozone concentration in a homogeneously mixed volume may be explained by looking at the ratio between \chem{NO_x} and VOCs, and identify the corresponding \chem{NO_x}-limited and VOC-limited regimes. In practice, transport and mixing complicates such approximation.


\subsection{Sources for Volatile organic compounds (VOCs)} \label{ssec:intro/ozone/voc}

	The definition of volatile organic compounds, or VOCs, varies. For consistency, we use the following definition from the United States Environmental Protection Agency (EPA):
	\begin{quotation}
		Volatile organic compounds (VOC) means any compound of carbon, excluding carbon monoxide, carbon dioxide, carbonic acid, metallic carbides or carbonates, and ammonium carbonate, which participates in atmospheric photochemical reactions \citep{EPA:2010ve}.
	\end{quotation}
	The extended definition also explicitly excludes specific chemicals for ``negligible photochemical reactivity,'' e.g. methane, ethane, CFCs, HCFCs, HFCs. Sources for VOCs can be partitioned largely into anthropogenic and biogenic emissions, with sporadic contributions from biomass burning and volcanic eruptions.

	\figuremacroW{Fig01-01_NEI-VOC}{National Emission Inventory (NEI) VOC emission trend}{\label{fig:epa-nei-voc} Trend for total NEI VOC emission for 1970--2012. Data points prior to 1990 were reported every 5 years. 2006 and 2007 are interpolation of 2005 and 2008 data. After 2008, except for transportation, all other emissions were assumed constant. While some changes are real, some are due to changes in inventory methodology.}{1.0}

	According to the EPA's 2008 National Emission Inventory (NEI), the leading source for anthropogenic VOC are industrial processes, emitting $6.9\times10^6$ tons and contributing to a third of the national anthropogenic VOC budget including wild fires (Fig. \ref{fig:epa-nei-voc}). VOC emission from transportation was $5.6\times10^6$ tons (27\% of total). These numbers are to be compared to 2005 figures of $7.4\times10^6$ tons (36\% of total) for industrial processes and $6.9\times10^6$ tons (34\%) for transportation. Further reduction in emission from transportation is seen in 2011, down to $4.3\times10^6$ tons. However, emission from wild fires have drastically increased to $2.8\times10^6$ tons (4\%) in 2008 from $1.9\times10^6$ tons (3\%) in 2005. Also, emission from the ``miscellaneous'' category, which includes sources such as prescribed fire and gas stations, also increased from $3.3\times10^6$ tons (9\%) in 2005 to $4.7\times10^6$ tons (14\%) in 2008. Totals excluding wild fire were $18.3\times10^5$ tons for 2005 and $17.7\times10^6$ tons for 2008.

	NEI2008 also estimated $31.7\times10^6$ tons of biogenic VOCs (BVOCs) in contiguous United States (CONUS). In rural areas, BVOCs naturally dominates. On the synoptic scale, as anthropogenic VOC emission is being reduced through regulatory efforts, the impact of BVOCs on tropospheric chemistry is also becoming more important. Of the hundreds of BVOCs identified, the global flux is dominated by isoprene (\chem{C_5H_8}) \citep{Guenther:2006kl}, which is primarily emitted by terrestrial foliage. Bottom-up global estimates of isoprene emission using Model for Emissions of Gases and Aerosols from Nature (MEGAN) based on leaf data gives about 600 Tg for 2003 \citep{Guenther:2006kl}, and its regional distribution has been further constrained with satellite observation of formaldehyde (\chem{HCHO}) column from satellite instruments such as the Global Ozone Monitoring Experiment (GOME) instrument on board of ERS-2 \citep{Palmer:2001nx, Palmer:2003cr,Palmer:2006qf},  the SCIAMACHY instrument on board of the ENVISAT  \citep{Dufour:2009fk, De-Smedt:2008uq, De-Smedt:2010kx}, and the Ozone Monitoring Instrument (OMI) on board of the EOS Aura satellite \citep{Millet:2008oq, Marais:2012kl}.

	Due to ozone's nonlinear sensitivity to changes in VOC, the simulation of ozone depends heavily on the accurate characterization of the emission inventory inputs or parameterization, which is the sole source for many hydrocarbons such as toluene or alkanes of high carbon numbers. Of particular interest to this study is how changes in each of these sources may impact the observed upper tropospheric ozone level and ultimately the radiative and chemistry budget.


\subsection{Anthropogenic sources for Nitrogen oxides (\anox)} \label{ssec:intro/ozone/nox}

	Due to the large contribution from BVOCs, the ambient ozone level is controlled by the oxides of nitrogen ($\chem{NO_x}\equiv\chem{NO}+\chem{NO_2}$), which is dominated by anthropogenic \chem{NO_x} (\anox). The straightforward conjecture is that the more \chem{NO_x} is available, the higher the ozone concentration. However, this is not the case in either ``clean'' air, wherein VOC level is low, or in the presence of high \chem{NO_x} level. Either case can induce \chem{NO_x}-titration, which is discussed briefly in \sect{intro/ozone}. In addition, lightning-generated \chem{NO_x}, or {\lnox} is also identified as an important natural source for atmospheric \chem{NO_x}. Introduction to {\lnox} is provided in the next section (\sect{intro/lightning}).

	\figuremacroW{Fig01-02_NEI-NOx}{National Emission Inventory (NEI) \chem{NO_x} emission trend}{\label{fig:epa-nei-nox} Trend for total NEI \chem{NO_x} emission for 1970--2012. Data points prior to 1990 were reported every 5 years. 2006 and 2007 are interpolation of 2005 and 2008 data. After 2008, except for transportation, all other emissions were assumed constant. While some changes are real, some are due to changes in inventory methodology.}{1.0}

	Global \chem{ANO_x} emission has gradually decreased over the last two decades. \citet{Muller:1992kx} estimated a 72\,\unit{Tg/yr} global emission in the early 1990's.  Intergovernmental Panel of Climate Change (IPCC) Third Assessment Report (AR3) estimated 28--32\,\unit{Tg}\,\chem{N} from fossil fuel combustion, out of an approximately 50 Tg N annual global total \chem{NO_x} sources at the end of the 20th century \citep[][ and references therein]{Schumann:2007fk,Lamarque:2010fk}. In particular, emission of {\anox} in the United States has been reduced substantially over the years. Figure \ref{fig:epa-nei-nox} shows the NEI national trend expressed in thousands of tons of total \chem{NO_x}. The primary sources of {\anox} in the U.S. have been consistently transportation and fuel combustion, respectively contributing to 63\% and 29\% of the total {\anox} Comparing 2005 and 2008 NEI, \chem{NO_x} from transportation has been reduced by 4.2\% and \chem{NO_x} from fuel combustion has been reduced by 21\%.

	Similar to VOCs (\ssect{intro/ozone/voc}), the reduction in {\anox} has gradually increased the relative importance of its natural counterparts. Nonetheless, the overall decline in {\anox} does not preclude the role of {\anox} in tropospheric ozone chemistry. Specifically, the co-location of {\anox} and various anthropogenic sources for VOCs continues to dominate the anthropogenic-factor of ozone chemistry due to the relatively-fast reaction rates prior to detraining into the free troposphere.


%
% Section:
% 	Lightning parameterization
%

\section{Parameterizing Lightning-generated \chem{NO_x} ({\lnox})} \label{sec:intro/lightning}

The global emission rate of {\lnox} is highly uncertain and largely unconstrained, thus introducing uncertainties in ozone climate simulations. The best current estimate for {\lnox} is $5\pm3$\,\unit{Tg}\,\chem{N} per year globally, which contributes to around 10\% of the total \chem{NO_x} emission \citep{Schumann:2007fk}. Despite being a smaller direct source than {\anox}, {\lnox} has a comparable potential for ozone production due to its direct emission in the free troposphere, where it has a much longer lifetime than in the boundary layer \citep{Lamarque:1996kx,Allen:2000uq}. \citet{Hauglustaine:2001mi} used Model for OZone And Related chemical Tracers (MOZART) to study the ozone produced from {\lnox} and found ozone enhancement of over 140\% at 250\,\unit{mb} in January above southern Atlantic, Africa, and south America. Similarly, \citet{Cooper:2007cr} also found that the North American Monsoon upper tropospheric ozone enhancement, the central topic of this thesis, is largely due to {\lnox}.

\subsection{Flash rate parameterization}\label{ssec:intro/lightning/flash}

	A common first step to parameterizing {\lnox} emission is to constrain the lightning flash count distribution. Many studies have attempted to reduce the spatiotemporal uncertainty of the said distribution through a combination of ground-based lightning detection instruments \citep[e.g.][]{Boccippio:2001ys,Hansen:2010fk}, satellite-borne instruments \citep[e.g.][]{Ushio:2001kx,Jourdain:2010tw,Martini:2011fk}, convective outflow measurements \citep[e.g.][]{Pickering:1998sh,Skamarock:2003mq}, chemistry transport models \citep[e.g.][]{Price:1997fk,Allen:2010fk,Allen:2012fk,Ott:2010lo}. Using the predicted flash rate, one may then deduce the {\lnox} emission and its distribution once provided a method for flash-to-emission conversion, which will be discussed in the next section (\ssect{intro/lightning/lnox}).

	Work by Bernard Vonnegut (1914--1997) laid down the theoretical foundation for a physical-based parameterization of lightning flash rate.  Based on the works of \citet{Vonnegut:1963aa} and \citet{Williams:1985fk}, \citet{Price:1992wb} constructed a lightning parameterization based on cloud-top height. The theoretical basis may be summarized as follow \citep[after][]{Boccippio:2002uq}.

	Conceptualize the electric structure of a thunderstorm as two tangential spherical charge volumes of radius $R$ and charge density $\pm\rho$. Compared to more recent studies, which point to tripole \citep{Williams:1989uq} or multipole \citep{Stolzenburg:1998fk} charge structures, a dipole approximation is an obvious but reasonable simplification by accounting for the dominating region for charge separation. Then, assume an aspect ratio of unity, {\ie} horizontal dimensions are similar in size to the vertical dimension. \citet{Boccippio:2002uq} noted the potential of an ``at best'' factor-of-two inaccuracy under this assumption. A third simplification asserts that the dipole is maintained through a generator current modeled as a net charge transport velocity. Finally, it is assumed that the variation of this current is small.
	Expressing the generator power as a product of the generator current and the electric potential of the dipoles, we have $P = I\Phi$. Applying the definitions of $I$ and $\Phi$ and applying Gauss' Law, \citet{Boccippio:2002uq} writes \footnote{Assuming different charge geometry can lead to a different scaling relationship, e.g. asserting 2D charge plates instead of spheres would lead to $P\sim\rho_{gen}V_QR^2z_dE_{crit}$ \citep{Boccippio:2002uq}.}
	\begin{equation}\label{eqn:boccippio-P}
		P \sim \varepsilon^{-1}\rho^2V_QAz_d^2
	\end{equation}
	where $V_Q$ is the charging velocity, $A=\pi R^2$ is simply the area, and $z_d=2R$ is the dipole separation. Hence a fundamental conclusion is arrived here stating that the geometric term dominates the generator power. If power dissipation by lightning varies monotonically with the generator power and that each flash dissipates a constant amount of energy, then flash rate $f$ is proportional to the generator power $P$ and we write
	\begin{equation}\label{eqn:boccippio-f}
		f \sim \rho^2V_QAz^2
	\end{equation}
	Here, $z_d$ has been simplified to storm height $z$. Three more final assumptions on storm properties can be made to make this relationship operationally viable. (1) Applying the assumption on the storm aspect ratio, $A$ is replaced with $z^2$. (2) Using $w$ as the best approximation for $V_Q$. (3) Charge density has negligible variability at least within the dominant dipole regions. These simplifications produce the following relationship.
	\begin{equation}\label{eqn:boccippio-f(w,z)}
		f\sim wz^4
	\end{equation}
	Compiling data from several studies between 1962 -- 1979, \citet{Williams:1985fk} observes a linear relationship between vertical velocity and cloud size for continental storms. This empirical relation allows one final simplification to Equation \ref{eqn:boccippio-f(w,z)}:
	\begin{equation}\label{eqn:w85-f(z)}
		f_c\sim z^5
	\end{equation}
	This scaling relationship has been shown to be operationally viable in several studies \citep[e.g.][]{Yoshida:2009vn, Ushio:2001kx}. Compiling data from \citet{Williams:1985fk}, the empirical relationship of flash rate and cloud top height for continental storm is written as \citep[][hereafter PR92]{Price:1992wb}
	\begin{equation}\label{eqn:pr92-fc(z)}
		f_c(z) = 3.44\times10^{-5}z^{4.9}
	\end{equation}
	Alternatively\footnote{\citet{Baker:1995uq} theoretically derived $f_c\sim w^6$ instead.}, $f_c$ may be written as a function of $w$, giving $f_c\sim w^{4.54}$. While the linear relationship between vertical velocity and cloud top height in continental clouds is confirmed by \citet{Price:1992wb}, $w$ for marine clouds appear to scale much slower with cloud size:
	\begin{equation}\label{eqn:pr92-wm}
		w_m = 2.86z^{0.38}
	\end{equation}
	Subsequently, a separate parameterization for marine clouds is formulated as $6.2\times10^{-4}z^{1.79}$. Due to the lack of data, the marine relationship is not tested thoroughly, and it is later rewritten as $f_m=6.57\times10^{-6}z^{4.9}$ when effects of marine cloud condensation nuclei (CCN) is taken into account \citep{Michalon:1999fk}. Furthermore, the arrival to the PR92's marine formulation is shown to be formally inconsistent with \citet{Vonnegut:1963aa} and takes the form $f_c(z_m)$, {\ie} correlating continental flash rates with marine cloud top heights\citep{Boccippio:2002uq}.

	\citet{Price:1993fk} used the flash data from eleven states in the western United States, detected by wide-band magnetic direction finders, in combination with thunderstorm radar and radiosondes data to find a relation for the IC:CG ratio ($Z$) from cold-cloud depth ($d$), defined as the distance from freezing level to cloud-top.
	\begin{equation}\label{eqn:pr93-Z}
		Z = 0.021d^4-0.648d^3+7.49d^2-36.54d + 63.09
	\end{equation}
	
	In \citet{Price:1994fk}, a ``calibration factor'' ($c$) for the resolution dependency of PR92 is introduced by regridding 5 km data between 1983 and 1990 from the International Satellite Cloud Climatology Project data set \citep[ISCCP;][]{Rossow:1991aa} to different horizontal grid sizes. The resulting equation is as follow
	\begin{equation}\label{eqn:pr94-calib}
		c = 0.97241\exp(0.048203R)
	\end{equation}
	where $R$ is the grid area in squared degrees. \citet{Price:1994fk} claims that there is no dependence of $c$ on latitude, longitude, or season.

	Other bulk-scale or resolved-scale storm parameters may also be correlated with lightning flashes for the purpose of formulating alternative parameterization schemes. For instance, \citet{Allen:2002fk} and\citet{Allen:2010fk} implemented a parameterization of flash rate to the square of deep convective mass flux. \citet{Zhao:2009kx} and \citet{Choi:2005uq} based the flash rate prediction on both the deep convective mass flux and the convectively available potential energy (CAPE). \citet{Allen:2012fk} used a flash rate prediction scheme based on the convective precipitation rate. \citet{Petersen:2005fk} gave a linear relation between flash rate and ice water path (IWP). \citet{Deierling:2008uq} investigated a linear dependence of flash rate on updraft volume for $T<273\,\unit{K}$ and $w>5\,\unit{m\,s^{-1}}$. \citet{Hansen:2012aa} produced a lookup-table for flash rate from convective precipitation and mixed phase layer depth by correlating data from observations. \citet{Barthe:2010uq} compared several of these methods including PR92, through case studies, and showed that while the polynomial orders are lower in these formulations, the level of uncertainties may still be higher than PR92 due to a combination of errors from model biases in the parameters used, e.g. hydrometeors, and observational biases in the datasets used for constructing the relationships. \citet{Futyan:2007fk} arrived at a similar conclusion about the reduced reliability of precipitation-based approaches in global climate simulations for predicting lightning flash rate. Explicit schemes, wherein electrification within convective cells are resolved, also exist \citep[e.g.][]{Barthe:2005zr,Zhang:2003tp}. However, they are often too costly for synoptic modeling and thus their usages are limited to single-storm case studies.

\subsection{Lightning-generated \chem{NO_x}} \label{ssec:intro/lightning/lnox}


	Lightning-generated \chem{NO_x} ({\lnox}) is produced shortly following a channel formation. Two main theories exist in explaining such production. The first being the shockwave heating theory, which describes NO being formed in the air surrounding the lightning channel according to the Zel'dovich mechanism after being heated up by a cylindrical shock propagating outwards from the current \citep[][and references therein]{Borucki:1984dq}, which describes high \chem{NO} to \chem{N_2} equilibrium ratio begin maintained at that at the shock front after rapid cooling. However, \citet{Stark:1996bh} argued that the shockwave does not travel fast enough to produce the necessary temperature for Zel'dovich mechanism.  Using experimental data, \citet{Wang:1998ve} relates the production of \chem{NO} (in $10^{21}\,\unit{/m}$) to peak current ($I$) and pressure ($p$):
	\begin{eqnarray}
		n_{\chem{NO}}(I) &=& 0.14 + 0.026I + 0.0025I^2 \label{eqn:wang-n(I)} \\
		n_{\chem{NO}}(p) &=& 0.34 + 1.30 p \label{eqn:wang-n(p)}
	\end{eqnarray}
	The correlation factor ($r^2$) for the above fits to measurements are 0.89 and 0.67 respectively, but it should be noted that the coefficients in Equation \ref{eqn:wang-n(I)} have significant relative uncertainty ($\pm0.33$, $\pm0.036$, $\pm0.0009$ respectively).
	
	Using the mean peak current from remote sensing measurements, \citet{Price:1997fk} estimated that intracloud (IC) flashes produce only 10\% of the energy produced by cloud-to-ground (CG) flashes and thus deducing that the moles of \chem{NO} per flash from IC is also a factor of 10 smaller than that from CG, which is calculated at 1.1\,\unit{kmoles} per CG flashes. However, more recent studies have found that such ratio is grossly overestimated and that both IC and CG flashes produce approximately the same amount of \chem{NO}. \citet{Ott:2010lo} estimated 500\,\unit{moles}\,\chem{NO} per flash or 7\,\unit{kg}\,\chem{NO} per flash with a range of 360--700\,\unit{moles}\,\chem{NO} per flash. This is compared to a calculation using the median peak current from North American Lightning Detection Network \citep[NALDN;][]{Orville:2002uq} and relationship in \citet{Decaria:2000kl} after \citet{Wang:1998ve}.
	
	After a total number of \chem{NO} emission is estimated, one may want to assign a spatial distribution to the molecules. In models that are capable of resolving explicit electrification, flash path, and stroke tortuosity \citep[e.g.][]{Mansell:2002kx}, such details may be leveraged in combination of Equation \ref{eqn:wang-n(p)} to compute the vertical distribution \citep{Barthe:2007fk}. In regional models, a fixed column profile based on cloud extent or observations are often used instead. For example, \citet{Decaria:2000kl} prescribes a Gaussian-like distribution for emissions from CG flashes and a bimodal distribution for IC flashes. \citet{Pickering:1998sh} derived total {\lnox} profiles with a peak at the detrainment level and a maximum in the boundary layer by using measurements from multiple campaigns and 2D models. \citet{Ott:2010lo} corrected these profiles by using a 3D model, and found that convective updraft removes the boundary layer maximum and places a single maximum at 6--9 \,\unit{km} for subtropical and midlatitude storms. \citet{Hansen:2010fk} investigated the vertical distribution of flash segments for four storms and found that profiles can exhibit large variabilities.
	
\quad

\noindent In conclusion, lightning parameterization remains a major uncertainty in tropospheric ozone modeling. Many methods have been proposed with varying degrees of generality and complexity. This study has elected to use the simplest method available for flash rate parameterization, {\ie} \citet{Price:1992wb}, which is evaluated in depth in Chapter \ref{ch:lightning}.