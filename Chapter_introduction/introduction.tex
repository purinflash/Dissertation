\chapter{Introduction} \label{ch:introduction}

\ifpdf
    \graphicspath{{Chapter_introduction/figures/PNG/}{Chapter_introduction/figures/PDF/}{Chapter_introduction/figures/}}
\else
    \graphicspath{{Chapter_introduction/figures/EPS/}{Chapter_introduction/figures/}}
\fi

Upper tropospheric ozone has significant impacts on the radiative and chemical budgets of the atmosphere \citep{Kiehl:1999uq}. Global tropospheric ozone burden has seen an increase of 71--130\,\unit{Tg} since preindustrial period, with much of the uncertainties coming from the estimation of preindustrial emission scenarios for anthropogenic, biomass burning, and lightning sources \citep[][and references therein]{Lamarque:2005gb}. The radiative forcing resulting from this increase depends strongly on the vertical distribution and is the most sensitive near the tropopause \citep{Lacis:1990fk}. \citet{Gauss:2006zr} calculated that the global net radiative forcing resulting from a 7.9--13.8\,\unit{DU} increase in tropospheric ozone (ignoring stratospheric changes) varies between 0.25 and 0.45\,\unit{W\,m^{-2}}. In particular, several models used in the same study as well as \citet{Stevenson:1998fk} estimated up to 0.5--1.1\,\unit{W \,m^{-2}} increase over parts of northern mid-latitudes such as North America and the Mediterraneans. The ozone increases over these regions are also verified by ozonesonde records spanning periods between 1970's and 2004 \citep{Oltmans:2006kc}.

% Lamarque:2005 - trop ozone evolution since pre-industrial. Summary of other studies. 77 -- 130 Tg
% Stevenson:1998 - radiative forcing due to trop ozone increase. max 30 ppbv
% Lacis:1990 - vertical distribution. Most sensitive at tropopause
% Gauss:2006 - multi-model study showing 0.25--0.45 W/m2 increase in global net radiative forcing
% Oltmans:2006 - global ozonesonde study verifying northern mid-lat ozone increase

Recent studies have identified monsoonal ozone precursor accumulations above North America \citep[][and references therein]{Li:2005ss,Cooper:2009nx}, Asia \citep{Park:2007bh,Worden:2009ve}, and equatorial Africa \citep{Bouarar:2011ly} during summertime in the upper troposphere. \citet{Cooper:2007cr} calculated a median tropospheric ozone mixing ratio of 87~\unit{ppbv} (after removing stratospheric intrusion) above Huntsville,~AL in August 2006. This is about twice the value measured along the United States west coast during the same period. The observed upper tropospheric ozone enhancement is linked to the North American Monsoon anticyclonic circulation, which traps ozone precursors that subsequently enhances ozone production \citep{Li:2005ss}. \citet{Cooper:2006dq} estimated that up to 84\% of the enhancement observed above southern United States can be attributed to {\insitu} ozone production from lightning-produced \chem{NO_x} (L\chem{NO_x}). The radiative forcing of this enhancement is +0.50\,\unit{W\,m^{-2}}, 70\% or which attributable to enhancement through L\chem{NO_x}.

By leveraging improved lightning parameterization, online budgeting diagnostics, and passive tracers, this thesis intends to extend the result from previous studies and understand various contributing factors that led to the observed ozone distribution. The primary goals of this study are as follow:

\begin{enumerate}
\item{} Simulate the observed ozone distribution using a regional chemistry transport model and evaluate the outputs against independent observations;
\item{} Utilize online budgeting diagnostics and passive tracers to understand the evolution of the simulated ozone distribution;
\item{} Conduct sensitivity simulations to evaluate how the ozone enhancement would react to different emission scenarios from anthropogenic, biogenic, and lightning sources;
\item{} Evaluate outputs from a multiyear simulation to understand the inter-annual variability of the ozone enhancement and potential impacts from future emission scenarios.
\end{enumerate}