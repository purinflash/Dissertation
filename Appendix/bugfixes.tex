\chapter{WRF-Chem bug fixes} \label{apdx:bug}

\ifpdf
    \graphicspath{{Appendix/figures/PNG/}{Appendix/figures/PDF/}{Appendix/figures/}}
\else
    \graphicspath{{Appendix/figures/EPS/}{Appendix/figures/}}
\fi

Due to recent compiler switch, numerous bugs surfaced and caused the model to crash or produce undesirable results. This appendix details the bugs discovered and their corresponding fixes.

\section{Fast TUV}\label{a-sec:bug/ftuv}

The fast Tropospheric Ultraviolet-Visible (fast TUV, or FTUV) photolysis scheme is a simplified version of the widely used TUV scheme by \citet{Madronich:1987uq}. Details of FTUV parameterization of TUV can be found in \citet{Tie:2003ve}. Since the use of MOZART chemistry options triggers an alternative method for computing model top ozone column, this bug only concerns WRF-Chem simulations using non-MOZART chemistry options.

\subsection{Bug description}\label{a-ssec:bug/ftuv/bug}
To simulate the radiative transfer of the full atmosphere, model column ozone is supplemented with exo-model ozone column density, or ``exocolden.'' This is done during the call to the subroutine \textbb{photo\_inti} within \textbb{ftuv\_init}. One of the intended functions of \textbb{photo\_inti} is to add exocolden data every 2\,\unit{km} on top of the model top up to 50\,\unit{km} using reference data (\textbb{o3ref}). However, the model top input for this function has been hard-coded to 20\,\unit{km} regardless of the actual model top level height. The model set-up used in this study has a model top of 10\,\unit{hPa}, or $\sim30\,\unit{km}$. Therefore, the effect height profile becomes $0,\ldots,30,22,24,26,\ldots,48,50\,\unit{km}$ plus $2.9745\times10^{16}\,\unit{molecules\,cm^{-2}}$ above 50\,\unit{km}. Furthermore, before the calculation of slant column density, the total column ozone is also scaled to a fixed value (265\,\unit{DU}). However, this value is grossly under-representing the standard ozone column for the United States, which has a recommended value of 349\,\unit{DU} instead (\textit{S. Madronich}, personal communication, 2013). These bugs effectively have two primary consequences\footnote{The expected low bias caused by the 265\,\unit{DU} is largely offset by the larger distribution of ozone above the model top due to the extra 20--30\,\unit{km} layer.}:
\begin{enumerate}
	\item Since ozone is typically highest between 20--30\.\unit{km}, two ozone maxima are introduced, one located outside the model physical levels. Because of an extra ozone maximum above model top, older simulations with a similar model top (30\,\unit{km}) may see significantly lower photolysis rates above 20\,\unit{km}.
	\item When calculating the slant column density, the logarithm of \textbb{o2col} (depends on \textbb{dz}) is calculated in the subroutine \textbb{schu}. This effectively produces a \textbb{NaN} (Not-a-Number) at the model top when the heights drop from 30\,\unit{km} back down to 20\,\unit{km}. The is later used to index a look-up table. Since no index is initialized for the model top due to NaN value, it may either crash the model or produce unpredictable result depending on what the pre-existing value in the corresponding memory location was. Older runs may see missing photolysis rates at their model tops as a result.
\end{enumerate}

\subsection{Changes made}\label{a-ssec:bug/ftuv/fix}
	\begin{wraptable}{o}{0.5\textwidth}
	\vspace{-.5in}
		\begin{singlespacing}
		\caption{Compaison of photolysis rates from FTUV and TUV.}
		\centering
		\begin{tabular}{p{0.1\textwidth}|p{0.16\textwidth}p{0.16\textwidth}} \hline
		$z$ (\unit{km}) & \multicolumn{2}{c}{(FTUV-TUV)/TUV} \\ 
		 & \multicolumn{1}{c}{$J(\chem{O_3})$} & \multicolumn{1}{c}{$J(\chem{NO_2})$} \\ \hline\hline
		0	& +2.74\% & -11.8\% \\
		2	& +4.51\% & -9.88\% \\
		5	& +10.5\% & -9.63\% \\
		7	& +13.7\% & -5.47\% \\
		10	& +14.9\% & -3.00\% \\
		20	& -8.64\% & -1.39\% \\
		29	& -7.68\% & -1.90\% \\ \hline
		\end{tabular}
		\label{tab:bug/ftuv}
		\end{singlespacing}
	\vspace{-.3in}
	\end{wraptable}

The model top input for \textbb{photo\_inti} has now been changed to 30\,\unit{km}, the mean cloud top for the model set-up used in this study. The ozone column scaling has also been changed to 349\,\unit{DU}. Due to an overall low bias in stratospheric ozone in WRF-Chem, the ozone profile between 20--30\,\unit{km} (model levels) internal of FTUV has also been replaced by that from \textbb{o3ref}. With these changes, the FTUV-TUV comparison\footnote{The TUV code used here is an independent utility developed and implemented by Dr. Sasha Madronich at NCAR-ACD.} is shown in Table \ref{tab:bug/ftuv} for clear-sky condition with a solar zenith angle of $8.428^\circ$ with a surface albedo of 10\%. The remaining biases may be attributed to differences in actual ozone profiles used between WRF-Chem FTUV and the TUV code. It should also be noted that $J(\chem{NO_2})$ has slightly different cross-section int the two implementations, so differences are expected.

\subsection{Miscellaneous problems}\label{a-ssec:bug/ftuv/misc}
\begin{enumerate}
	\item Diagnostic outputs recently added (v3.4.0+) for photolysis spectral and rate information (\textbb{radfld}, \textbb{adjcoe}, \textbb{prate}) are copied through \textbb{nz}, which is the number of FTUV model levels including exocolden levels (59 in this study). However, these variables were only defined for \textbb{nref}($:=51$) levels. Thus for simulations where \textbb{nz}$>51$, this statement is a guaranteed crash or out-of-bound access. This problem is patched for this study by removing the respective copy statements since the diagnostics are not used.
	\item The surface albedo over sea water uses different data sets for MOZART and non-MOZART options. The concerning factor here is that the prescribed albedos is about an order of magnitude apart (e.g. 0.0747 for MOZART and 0.8228 for non-MOZART for the first spectral bin). No fix has been performed for this particular inconsistency due to lack of knowledge for the true value.
	\item Photolysis cross sections and top-of-atmosphere solar flux have not been updated to more recent recommendations since FTUV's initial implementation in 2003. No direct changes have been made to these values, but constant scaling factors ($\alpha$) have been applied to several photolysis reactions that displayed significant biases during CALNEX 2010 flight measurements (\textit{C. Knote}, personal communication, 2013). These reactions include $\alpha(\chem{HNO_4})=0.35$, $\alpha(\chem{MGLY})=0.2$, and $\alpha(\chem{GLY})=0.14$, and $\alpha(\chem{H_2O_2})=1.4$.
\end{enumerate}

\section{Passive tracer-convective transport}\label{a-sec:bug/tracer}