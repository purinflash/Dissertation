\newcommand{\sect}[1]{Section~\ref{sec:#1}}
\newcommand{\ssect}[1]{Section~\ref{ssec:#1}}
\newcommand{\fig}[1]{Figure~\ref{fig:#1}}

\newcommand{\diff}[2]{\frac{\partial #1}{\partial #2}}
\newcommand{\diffr}[1]{\diff{#1}{r}}
\newcommand{\diffth}[1]{\diff{#1}{\theta}}
\newcommand{\diffz}[1]{\diff{#1}{z}}
\newcommand{\degr}{$^\circ$}

\newcommand{\vth}{V_{\theta}}

\newcommand{\insitu}{\textit{in situ}}
\newcommand{\ie}{\textit{i.e.}}

\newcommand{\lnox}{L\chem{NO_x}}
\newcommand{\anox}{A\chem{NO_x}}

\newcommand{\resolution}[1]{#1\,\unit{km}\,$\times$\,#1\,\unit{km}}

\newcommand{\twochoices}[2]{\left\{ \begin{array}{lcc}
        \displaystyle #1 \\ \vspace{-10pt} \\
        \displaystyle #2 \end{array} \right. } %}

\newcommand{\threechoices}[3]{\left\{ \begin{array}{lcc}
        #1 \\ #2 \\ #3 \end{array} \right. }    %}

\newcommand{\fourchoices}[4]{\left\{ \begin{array}{lcc}
        #1 \\ #2 \\ #3 \\ #4 \end{array} \right. }      %}

\newcommand{\twovec}[2]{\left(\begin{array}{c} #1 \\ #2 \end{array}\right)}
\newcommand{\threevec}[3]{\left(\begin{array}{c} #1 \\ #2 \\ #3 \end{array}\right)}
\newcommand{\twomatrix}[4]{\left(\begin{array}{cc} #1 & #2 \\ #3 & #4 \end{array}\right)}

% From Copernicus class file

\newcommand{\tophline}{\hline\noalign{\vspace{1mm}}}
\newcommand{\middlehline}{\noalign{\vspace{1mm}}\hline\noalign{\vspace{1mm}}}
\newcommand{\bottomhline}{\noalign{\vspace{1mm}}\hline}
%% \hhline is obsolete and only kept for compatibility:
\newcommand{\hhline}{\noalign{\vspace{1mm}}\hline\noalign{\vspace{1mm}}}
%%
%% definition of \vec
%%
\DeclareRobustCommand*{\vec}[1]{\ensuremath{%
\mathchoice{\mbox{\boldmath$\displaystyle#1$}}
           {\mbox{\boldmath$\textstyle#1$}}
           {\mbox{\boldmath$\scriptstyle#1$}}
           {\mbox{\boldmath$\scriptscriptstyle#1$}}}}
\def\testbx{bx}
%%
%% definition of \chem and \unit
%%
\if@hvmath
   \DeclareRobustCommand*{\chem}[1]{\ensuremath{%
   \mathcode`\-="0200\mathcode`\=="003D% no space around "-" and "="
   \ifx\testbx\f@series\mathbf{#1}\else\mathrm{#1}\fi}}
   \DeclareRobustCommand*{\unit}[1]{\ensuremath{\def\mu{\mbox{\textmu}}\def~{\,}%
   \ifx\testbx\f@series\mathbf{#1}\else\mathrm{#1}\fi}}
\else
   \let\mathrm\mathsf \let\rm\sf
   \DeclareRobustCommand*{\chem}[1]{\ensuremath{%
   \mathcode`\-="0200\mathcode`\=="003D% no space around "-" and "="
   \ifx\testbx\f@series\mbox{\boldmath$\mathsf{#1}$}\else\mathsf{#1}\fi}}
   \DeclareRobustCommand*{\unit}[1]{\ensuremath{\def\mu{\mbox{\textmu}}\def~{\,}%
   \ifx\testbx\f@series\mbox{\boldmath$\mathsf{#1}$}\else\mathsf{#1}\fi}}
\fi   

% Other macros

\newcommand{\nox}{\chem{NO_x}}
\newcommand{\hox}{\chem{HO_x}}

\widowpenalty10000
\clubpenalty10000


% insert a centered figure with caption and description
% parameters 1:filename, 2:title, 3:description and label
\newcommand{\figuremacro}[3]{
	\begin{figure}[htbp]
		\centering
		\includegraphics[width=1\textwidth]{#1}
		\caption[#2]{\textbf{#2} --- #3}
		\label{#1}
	\end{figure}
}

% insert a centered figure with caption and description AND WIDTH
% parameters 1:filename, 2:title, 3:description and label, 4: textwidth
% textwidth 1 means as text, 0.5 means half the width of the text
\newcommand{\figuremacroW}[4]{
	\begin{figure}[htbp]
		\centering
		\includegraphics[width=#4\textwidth]{#1}
		\caption[#2]{\textbf{#2} --- #3}
		\label{#1}
	\end{figure}
}

% inserts a figure with wrapped around text; only suitable for NARROW figs
% o is for outside on a double paged document; others: l, r, i(inside)
% text and figure will each be half of the document width
% note: long captions often crash with adjacent content; take care
% in general: above 2 macro produce more reliable layout
\newcommand{\figuremacroN}[3]{
	\begin{wrapfigure}{o}{0.5\textwidth}
		\centering
		\begin{singlespacing}
		\includegraphics[width=0.48\textwidth]{#1}
		\caption[#2]{{\small\textbf{#2} --- #3}}
		\label{#1}
		\end{singlespacing}
	\end{wrapfigure}
}

% predefined commands by Harish
\newcommand{\PdfPsText}[2]{
  \ifpdf
     #1
  \else
     #2
  \fi
}

\newcommand{\IncludeGraphicsH}[3]{
  \PdfPsText{\includegraphics[height=#2]{#1}}{\includegraphics[bb = #3, height=#2]{#1}}
}

\newcommand{\IncludeGraphicsW}[3]{
  \PdfPsText{\includegraphics[width=#2]{#1}}{\includegraphics[bb = #3, width=#2]{#1}}
}

\newcommand{\InsertFig}[3]{
  \begin{figure}[!htbp]
    \begin{center}
      \leavevmode
      #1
      \caption{#2}
      \label{#3}
    \end{center}
  \end{figure}
}
