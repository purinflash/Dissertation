\title{Upper Tropospheric Ozone Enhancement during the North American Monsoon
evaluated using the Weather Research and Forecasting Model with Chemistry (WRF-Chem)}

\author{J.}{Wong}

\otherdegrees{
		M.S., University of Colorado, 2010 \\
		M.A., University of Arkansas, 2007 \\
		B.S., University of Arkansas, 2006 }

\degree{Doctor of Philosophy}{Ph.D., Atmospheric and Oceanic Sciences}
\degreeyear{2013}
\dept{Department of}{Atmospheric and Oceanic Sciences}

\advisor{}{Dr.~David Noone}			%  #2 {name}
\reader{Dr.~Darin Toohey}

%\reader{Dr.~Mary Barth}
%\readerThree{Dr.~Mary Barth}		%  3rd person to sign thesis
%\readerFour{Dr.~Bill Skamarock}
%\readerFive{Dr.~Jana Milford}

\abstract{  \OnePageChapter
	Upper tropospheric ozone has significant impacts on the total atmospheric chemistry
	and radiative budget. Previous studies noted an upper tropospheric ozone enhancement above
	southern United States during the North Ameican Monsoon (NAM). This recurring phenomenon
	has been observed by the satellite-borne Tropospheric Emission Spectrometer (TES) and
	IONS-06 ozonesondes. Using the Weather Research and Forecasting model with Chemistry
	(WRF-Chem), we attempt to simulate the ozone enhancement and understand the underlying
	structure, chemical pathways, and sensitivity to emissions.
	
	Using a modified lightning parameterization based on the Price and Rind scheme, a July-August simulation is performed using WRF-Chem.
	Validation shows that flash rate is over-predicted by a factor of 10, which subsequently causes
	an overestimation of \chem{O_3}. Despite the amplified ozone
	enhancement, boundary layer and stratospheric tracers do not show substantial
	differences within and outside the anticyclone, contrary to what has been predicted in other studies. On the other hand, lightning tracers, lateral
	boundary tracers, tracer-tracer correlations, and chemistry pathways can be used to distinguish
	between the anticyclone and the surrounding area.
	
	Sensitivity and control simulations are also performed to investigate how different emission sources
	contribute to the ozone enhancement. It is shown that lightning emission enhances \chem{NO_x} superlinearly,
	which leads to \chem{NO_x}-titration in the upper troposphere. On the other hand, anthropogenic emission is shown to
	modify the ozone concentration and chemistry profiles via sensitivity to VOC, and biogenic emission
	affects ozone via sensitivity to \chem{NO_x}. As a result of the changes in ozone mixing ratio profile,
	convective tendency is also affected, which forms a nonlinear feedback between chemistry and convection
	in response to perturbation in emissions.
}

%\dedication[Dedication]{	% NEVER use \OnePageChapter here.
%\begin{center}
%	To procrastination \\
%	For I shall write this later.
%\end{center}}

\acknowledgements{	\OnePageChapter	% *MUST* BE ONLY ONE PAGE!
	First and foremost, I would like to thank Dr. Mary Barth (NCAR/ACD) and Dr. David Noone
	(CIRES/ATOC) for providing the opportunities and exceptional guidance throughout my
	graduate study at University of Colorado at Boulder. This work was supported by the NASA
	Atmospheric Composition program (grant number NNH07AM47G) and appointment via the
	NRCM project (NSF-EaSM grant AGS-1048829; ``Developing a Next-Generation Approach to
	Regional Climate Prediction at High Resolution''). The NLDN dataset is provided by
	Vaisala and made available via Dr. James Crawford of NASA Langley Research Center
	and Dr. Owen Cooper at NOAA/ESRL. The ENTLN dataset is prepared and provided by Steve
	Prinzivall at Earth Networks. We would like to ackowledge high-performance computing
	support provided by NCAR's Computational and Information Systems Laboratory, sponsored 
	by the National Science Foundation (project code P35071370). We also thank the following
	individuals who have provided valuable advices and insights in relation to this study and assistance
	in developing the diagnostics and lightning parameterization modules for WRF-Chem:	
	Dr. Linnea Avallone (LASP/ATOC), Dr. Georg Grell  (NOAA/ESRL/GSD), Dr. Sasha Madronich
	(NCAR/ACD), Dr. David Noone (CIRES/ATOC), Dr. Steven Peckham (NOAA/ESRL/GSD), Dr.
	Gabriele Pfister (NCAR/ACD), Dr. Bill Skamarock (NCAR/MMM), Dr. John Worden (NASA/JPL).
	Finally, I would like to acknowledge and thank Dr. Tom Warner (NCAR/RAL, CU-Boulder/ATOC)
	for his support before he passed away on May 30, 2011.
	}

%\ToCisShort	% use this only for 1-page Table of Content
%\LoFisShort	% use this only for 1-page Table of Figures
%\LoTisShort	% use this only for 1-page Table of Tables