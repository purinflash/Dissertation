\title{Upper Tropospheric Ozone Enhancement during North American Monsoon using Regional Chemistry Transport Models}

\author{J.}{Wong}

\otherdegrees{B.S., University of Arkansas, 2006 \\
		M.A., University of Arkansas, 2007 \\
		M.S., University of Colorado, 2010 }

\degree{Doctor of Philosophy}{Ph.D., Atmospheric and Oceanic Sciences}
\degreeyear{2013}
\dept{Department of}{Atmospheric and Oceanic Sciences}

\advisor{Dr.}				%  #1 {title}
	{David Noone}			%  #2 {name}

\reader{Dr.~Linnea Avallone}		%  2nd person to sign thesis
\readerThree{Dr.~Mary Barth}		%  3rd person to sign thesis
\readerFour{Dr.~Bill Skamarock}
\readerFive{Dr.~Jana Milford}

\abstract{  \OnePageChapter
	(Tentative abstract) The upper-troposphere/lower-stratosphere (UTLS) ozone budget has a significant impact on the total atmospheric chemistry and radiative budget. Previous studies noted a UT positive ozone anomaly above southern United States, which was also observed by the satellite-borne Tropospheric Emission Spectrometer (TES). Using the Weather Research and Forecasting coupled with chemistry (WRF-chem), a short-term budget has been constructed for the ozone concentration within the region of the anomaly at about 160hPa during the week prior to August 23, 2006. Major contributing sources of variance have been identified as chemistry, advection, convection, and dry vertical mixing. Among these identified source terms, advective tendencies dominate the local (180km x 180km) short-term ozone variance. The WRF-chem results also suggest that direct convective transport of ozone from the boundary layer dilutes UT concentration. However, boundary layer chemicals lofted by the same convective mechanism may later enhance the UT ozone content. From this budget, one can identify key components that can improve the skill in simulating the UTLS ozone anomaly during a North American Monsoon.
}

\dedication[Dedication]{	% NEVER use \OnePageChapter here.
\begin{center}
	To procrastination \\
	For I shall write this later.
\end{center}}

\acknowledgements{	\OnePageChapter	% *MUST* BE ONLY ONE PAGE!
	acknowledgement here
	}

%\ToCisShort	% use this only for 1-page Table of Content
%\LoFisShort	% use this only for 1-page Table of Figures
%\LoTisShort	% use this only for 1-page Table of Tables