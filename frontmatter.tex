\title{Upper Tropospheric Ozone Enhancement during North American Monsoon
evaluated using the Weather Research and Forecasting Model with Chemistry (WRF-Chem)}

\author{J.}{Wong}

\otherdegrees{
		M.S., University of Colorado, 2010 \\
		M.A., University of Arkansas, 2007 \\
		B.S., University of Arkansas, 2006 }

\degree{Doctor of Philosophy}{Ph.D., Atmospheric and Oceanic Sciences}
\degreeyear{2013}
\dept{Department of}{Atmospheric and Oceanic Sciences}

\advisor{Dr.}				%  #1 {title}
	{David Noone}			%  #2 {name}
\reader{Dr.~Mary Barth}
%\reader{Dr.~Darin Toohey}		%  2nd person to sign thesis
%\readerThree{Dr.~Mary Barth}		%  3rd person to sign thesis
%\readerFour{Dr.~Bill Skamarock}
%\readerFive{Dr.~Jana Milford}

\abstract{  \OnePageChapter
	The upper tropospheric ozone budget has significant impacts on the total atmospheric chemistry
	and radiative budget. Previous studies noted an upper tropospheric ozone enhancement above
	southern United States during the North Ameican Monsoon (NAM). This recurring phenonmenon
	has been observed by the satellite-borne Tropospheric Emission Spectrometer (TES) and the
	IONS-06 ozonesondes. Using the Weather Research and Forecasting model with Chemistry
	(WRF-Chem), we attempt to simulate the ozone enhancement and understand the underlying
	structure, chemical pathways, and sensitivity to emissions.
	
	The Price and Rind lightning parameterization based on cloud-top height is a commonly used
	method for predicting flash rate in global climate models. To understand its behavior at the resolution
	used in our study, we tested the flash rate parameterization, the intra-cloud/cloud-to-ground
	(IC\,:\,CG) partitioning parameterization, and the associated resolution dependency ``calibration
	factor'' by Price and Rind at multiple resolutions. We showed that skills in predicting flash rate
	varies accordingly with model biases in convection, but an erroneous drop-off in the frequency
	distribution is generated by the IC\,:\,CG partitioning and that the resolution dependency factor
	is insufficient in accounting for changes in resolution below 36\,\unit{km}.
	
	Using a modified PR92 parameterization, a July-August simulation is performed using WRF-Chem.
	Validation shows that flash rate is over-predicted by a factor of 10, which subsequently causes
	an overestimation of \chem{NO_x}, \chem{CO}, and \chem{O_3}. Despite the amplified ozone
	enhancement, mixing ratios, boundary layer, and stratospheric tracers do not show substantial
	differences within and outside the anticyclone. On the other hand, lightning tracers, lateral
	boundary tracers, tracer-tracer correlations, and chemistry pathways can be used to distinguish
	between the anticyclone and the surrounding area.
	
	Sensitivity and control simulations are performed to investigate how different emission sources
	contribute to the ozone enhancement. It is shown that the responds of ozone to lightning-generated
	\chem{NO_x} is nonlinear between tuned ($0.1\times$) and untuned ($1\times$) lightning scenarios.
	Examining vertical profiles of chemical tendencies and fractions of radicals terminated through
	\chem{NO_x}, we found that \chem{NO_x}-titration is occurring in the upper troposphere as a
	result of the super-linear increase of \chem{NO_x} with higher lightning emission rate.
	Perturbing from the $1\times$ scenario, anthropogenic emission is shown to control upper
	tropospheric ozone via sensitivity to VOC, and biogenic emission impacts the ozone level via
	sensitivity to \chem{NO_x}. As a result of the enhanced ozone production (and loss through
	\chem{NO_x}-titration), vertical profiles of mixing ratios are also significantly modified, thus
	affecting convective tendencies. These changes, coupled with the effect of convective transport
	of ozone precursor, forms a nonlinear feedback between chemistry and convection in response
	to perturbation in emissions as opposed to the simpler one-way model between the two processes.
}

\dedication[Dedication]{	% NEVER use \OnePageChapter here.
\begin{center}
	To procrastination \\
	For I shall write this later.
\end{center}}

\acknowledgements{	\OnePageChapter	% *MUST* BE ONLY ONE PAGE!
	acknowledgement here
	}

%\ToCisShort	% use this only for 1-page Table of Content
%\LoFisShort	% use this only for 1-page Table of Figures
%\LoTisShort	% use this only for 1-page Table of Tables