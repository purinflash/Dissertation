\chapter{Summary} \label{ch:summary}

\ifpdf
    \graphicspath{{5_summary/figures/PNG/}{5_summary/figures/PDF/}{5_summary/figures/}}
\else
    \graphicspath{{5_summary/figures/EPS/}{5_summary/figures/}}
\fi

Ozone in the troposphere acts as a greenhouse gas and causes harm to the ecosystem.
The recurring upper tropospheric ozone enhancement during the North American Monsoon has
been shown to perturb upper tropospheric ozone by more than 20\,\unit{ppbv} above background values,
which leads to a radiative forcing up to 0.50\,\unit{W\,m^{-2}} \citep{Li:2005ss,Cooper:2007cr,Choi:2009bh}.
Of all the meteorological and chemical processes that contributes to the ozone enhancement,
lightning-generated nitrogen oxides (lightning \chem{NO_x}, or {\lnox}), is shown to be the most
prominent factor, which contributes 25--30\,\unit{ppbv} of the ozone maximum at
250\,\unit{hPa}\citep{Cooper:2007cr}. It is important to note that {\lnox} emission is largely unconstrained
with a global annual estimate of $5\pm3\,\unit{Tg\,a^{-1}}$ \citep{Schumann:2007fk}. Despite the
uncertainty, only a few studies have investigated {\lnox} or lightning flash rate parameterization at the regional scale
\citep[e.g.][]{Allen:2010fk,Allen:2012fk,Barth:2012qf,Wong:2013vn}. Despite advancements in the
physical formulations for the parameterization \citep[][and references therein]{Barthe:2010uq},
the \citet[PR92;][]{Price:1992wb} method based on cloud top height continues to show
merits despite its own limitations \citep{Boccippio:2002uq}.

To apply PR92 to the 2006 case study, we have implemented PR92 into WRF-Chem
(Apdx~\ref{apdx:lnox-doc}). The performance of this implementation has been evaluated
\citep[][and Ch.~\ref{ch:lightning}]{Wong:2013vn} and the results are summarized as follow. It is
found that the statistics generated by a 3-month meteorological simulation are
comparable ($\sim2.4\times$ for 2006, $\sim1\times$ for 2011) to observations from NLDN and
ENTLN with a 2-\,\unit{km} cloud-top reduction, which is included to reconcile the differences between the level
of neutral buoyancy (LNB) cloud-top proxy and radar reflectivity cloud-top. This reduction is selected
to be an appropriate adjustment given a sufficiently large number of convective events. Thus, it
does not guarantee applicability for individual storm. However, the
IC\,:\,CG ratio produced by the \citet[][PR93]{Price:1993fk} method is shown to generate a erroneous
drop-off in the histogram and is thus deemed unreliable. Finally, it is shown that the resolution
dependency factor from \citet[][PR94]{Price:1994fk} is not applicable for a convective parameterized
model due to the interpretation of cloud-top heights within a grid. To reconcile differences
in model resolution, we suggest scaling by areal ratio to 36\,\unit{km}, at which convective
core density (number of core per grid) is expected to be close to unity. However, it should be noted
that the base resolution (36\,\unit{km}) may vary spatially or dependent or storm scale.

Due to differences in the simulated meteorology, flash rate is overestimated by a factor 10 in the
2006 case study, compared to the factor of 2 from the simulation evaluated in  Chapter~\ref{ch:lightning}.
The overpredicted {\lnox} emission subsequently leads to a more severe upper tropospheric
ozone enhancement. The ozone mixing ratio is simulated to be about 21\% higher than that
observed by TES. Similarly, \chem{CO} is shown to be high compared to TES due to lower
\chem{OH} as the result of excessive \chem{O_3} and \chem{NO_x}, but validation
against MOPITT shows negligibly low biases in the upper troposphere. Formaldehyde is validated against SCIAMACHY retrievals
from KNMI TEMIS. Because of the high uncertainties ($\sigma_N$) in the retrieved values,
majority of the data points lie within $\pm1\sigma$ despite low apparent correlations. On the
other hand, \chem{NO_2} is shown to be over-predicted for events $>2\times10^{15}\,\unit{molec.\,cm^{-2}}$,
which is expected due to excessive {\lnox}.

The differences between the chemistry within the anticyclone and that outside are evaluated
based on tracer-tracer correlations, passive tracer diagnostics, and tendency diagnostics. It
is concluded that contrary to previous studies, the NAM anticyclonic circulation does not
sufficiently increase the amount of boundary layer air detrained into the upper troposphere above
that in the surrounding area. On the other hand, the lack of influence from the air outside the anticyclone on the air inside, as quantified
by the tracer-tracer correlations and lateral boundary tracers, sufficiently distinguishes the
composition within the anticyclone from that outside.

To quantify the impact of model biases as well as the sensitivity of ozone to {\lnox} emissions,
additional simulations are performed without lightning and with reduced lightning tuned according
to the flash rate validation. It is shown that as a result of the super-linear response of upper
tropospheric \chem{NO_x} mixing ratios to {\lnox}, ozone production crosses the threshold
beyond which \chem{NO_x}-titration occurs, thus reducing and eventually reversing the
sensitivity at the detrainment level (150--250\,\unit{hPa}). Furthermore, changes in the ozone
vertical profile due to lightning emission also sufficiently modified the vertical gradients, which
subsequently affected the resulting convective tendencies, thus forming a nonlinear feedback
between convective transport and chemistry.

The contributions to the ozone enhancement from anthropogenic and biogenic emissions are also examined
through sensitivity experiments, wherein the respective emission sources are suppressed. In the
experiment pertaining anthropogenic emission, VOC, with \chem{CO} as the proxy, is observed
to be the controlling factor in the sensitivity. Due to \chem{NO_x}-titration, upper tropospheric
ozone production responds negatively to the increase in anthropogenic emission. On the other hand, biogenic
emission is responsible for \chem{NO_x}-losses in the upper troposphere (within the
\chem{NO_x}-titration regime), which subsequently allows higher net ozone production (again,
within the \chem{NO_x}-titration regime). It should be noted that this result is partially a consequence
of excessive {\lnox} within the model, which pushes the upper tropospheric chemistry into the \chem{NO_x}-titration
regime. A similar experiment is conducted by \citet{Li:2005ss} and found that enhancements in \chem{CO} as a result of
these emissions are the primary pathway of how these chemical sources contribute to the ozone enhancement.

In conclusion, and directly responding to the questions raised in Chapter~\ref{ch:introduction}:
\begin{enumerate}
	\item WRF-Chem is able to simulate the occurrence of the ozone enhancement despite an
	order of magnitude bias in \chem{NO_x} as the result of parameterizing lightning based on
	parameterized convection;
	\item In situ chemical production is the dominating tendency for the long term variability of
	the ozone enhancement after filtering out the advective component. However, contrary to
	results from previous studies \citep[e.g.][]{Li:2005ss,Cooper:2007cr,Barth:2012qf}, the
	anticyclone does not create a distinctive region wherein boundary layer air is aged above
	levels in the surrounding area;
	\item Lightning-generated \chem{NO_x} emission is responsible for 16\% increase in upper
	tropospheric ozone in the control and high-\chem{LNO_x} scenarios at 300\,\unit{hPa}.
	On the other hand, anthropogenic emission is responsible for $\sim6\%$ and biogenic
	emission is responsible for $\sim13\%$ with high {\lnox} emission.
\end{enumerate}

\section{Perspectives and outlook}

Despite the ozone enhancement being a relatively self-contained feature, its spatial structures
and chemical pathways are highly non-uniform and its response to perturbations in the emission
scenarios are often not monotonic. Thus, to obtain more definitive results, studies
focusing on specific scenarios and conditions are required to narrow down on specific chemical
or meteorological regimes. In order words, it is insufficient to simply define a process such as
convective transport and examine its link to the ozone enhancement. Furthermore, the initially
super-linear sensitivity to lightning-generated \chem{NO_x} also demands a finely tuned lightning
parameterization or one that is less sensitive to biases in the convective parameterization, which
is further complicated by the feedbacks between chemistry and convective transport. Several
potential improvements to this study and remaining questions follow.

Flash rate assimilation, as done by \citet{Cooper:2009nx}, can potentially
increase the fidelity of lightning \chem{NO_x} emission in the model and thus the ozone chemistry.
As total lightning monitoring instruments are being expanded or deployed(e.g. NLDN, ENTLN, GOES-R), and the
data becoming available, it is possible to revisit this approach again. However, because the vertical
distribution of {\lnox} emission has significant impacts on the resulting ozone chemistry
\citep{Pickering:1998sh}, it is also important to guarantee that convection and lightning colocate.
Thus, short simulations that re-initialize frequently may be the preferred mode of simulation as
opposed to season-long simulations for the flash-rate-assimilated approach to avoid model drift in meteorology.

While it has been shown that anthropogenic and biogenic emissions affect the ozone
enhancement through different mechanisms, the precise chemical pathways are yet to be
determined. Therefore, additional analyses aiming to understand the
differences in chemistry are recommended. Examples of the suggested analyses have already
been done for the sensitivity of ozone to {\lnox} emission (see Sect.~\ref{sect:sensitivity})
as well as for peroxy radical formation by \citet{Barth:2012qf}. However, the large number of
emitted species from anthropogenic and biogenic sources, and subsequently the number of
reactions involved, will undoubtedly complicate the analysis. Hence, it is recommended that
these studies be performed with minimal scope to maximize focus.