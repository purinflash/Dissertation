\ifpdf
    \graphicspath{{Chapter_2006/figures/PNG/}{Chapter_2006/figures/PDF/}{Chapter_2006/figures/}}
\else
    \graphicspath{{Chapter_2006/figures/EPS/}{Chapter_2006/figures/}}
\fi

\section{Discussion}\label{sec:2006/discussion}

\subsection{Tracer relations}\label{ssec:2006/discuss/tracer}

The relationship between ozone and carbon monoxide (\chem{CO}) has been used to study tropospheric-stratospheric transition in
chemical regimes \citep[e.g.][and references therein]{Pan:2007sw,Hegglin:2009fk}. Similarly, it may be applied to tropospheric data
alone to identify contributions from various pathways \citep[e.g.][]{Zhang:2006zr,Voulgarakis:2011fk,Cristofanelli:2013uq}. Ignoring
\chem{NO_x}, ozone is expected to be strongly positively correlated with \chem{CO} \citep{Chin:1994kx}. However, since the dominating
source for \chem{CO} is surface emission, with a limited lifetime, \chem{CO} decreases rapidly with height. At the same time, significant
contribution of stratospheric ozone and upper tropospheric production cause ozone to increase rapidly with height. Together, \chem{O_3-CO}
correlation over a sufficiently expansive column can expect to observe an "L"-shaped joint distribution.

\figuremacroN{tracer/tracer_correlations}{\chem{CO}-\chem{O_3} model correlation}{\label{fig:2006/o3co}
\textbf{(a)} Anticyclone ($\bar{Z}_{300}>9730\,\unit{m}$) \chem{CO}-\chem{O_3}(adjusted) joint-distribution separately
colored by pressure levels. \textbf{(b)} Same as \textbf{a} but outside the anticyclone between 9710\,\unit{m} and
9730\,\unit{m}. Data points south of 25$^\circ$N have been removed to reduce impact of the southern boundary. \vspace{-.2in}}

Here we correlated the August average model ozone and \chem{CO} from the surface up to 100\,\unit{hPa} (Fig.~\ref{fig:2006/o3co}.
Data points are first separated by within and outside of the anticyclone, defined by the boundary $\bar{Z}_{300}$ as previously
defined in Section~\ref{ssec:2006/gen/ozone}. Then, each panel is further classified by pressure level split at 300 and 700\,\unit{hPa}
to represent the lower, middle, and upper troposphere. To account for model biases, ozone VMRs between 100--300\,\unit{hPa}
is adjusted by the biases in mean and variance computed in Section~\ref{ssec:2006/gen/ozone}. No adjustment is performed at
the other pressure levels due to insignificant bias compared to observations. While the value shifted, the shape and features of the
joint-distribution remain consistent before and after the adjustment. Several features annotated in the figure are discussed below:

Feature A indicates a type of air mass exist only outside the anticyclone. This unique feature corresponds to subtropical Atlantic
influx from the southern boundary. It has slightly higher \chem{CO} but the ozone VMRs at 300\,\unit{hPa} from lower latitudes are
lower in general. These air masses intermittently contribute to the anticyclone, similar to the air masses mixed across the jet stream
from the north (see Sect.~\ref{ssec:2006/gen/ozone} \& \ref{ssec:2006/gen/co}). Feature B indicates the upper tropospheric branch
of the \chem{O_3}-\chem{CO} correlation. It is evident from the labels, which are placed at the same locations in both panels, that
\chem{CO} is more effective in raising the ozone level within the upper troposphere above the baseline, indicated by the dashed
line (feature C). Similar to the alphabetic labels, the baseline is also identical in both panels. Thus, it can be seen that the background
\chem{O_3}-\chem{CO} correlations are consistent both inside and outside the anticyclone. Finally, feature D denotes the mid-to-lower
tropospheric branch, which includes the boundary layer. Similar to B, D is chemically different across the anticyclone. Since circulation
is not a factor at low altitudes, the other plausible explanation is the abundant sources of anthropogenic and biogenic emissions of both
VOCs and \chem{NO_x}, which subsequently enhance in-situ ozone production within this layer and transported upwards.

\section{Sensitivity study}\label{sec:2006/sens}
\subsection{Anthropogenic emission}\label{ssec:2006/sens/anthrop}
\subsection{Biogenic emission}\label{ssec:2006/sens/bio}
\subsection{Lightning emission}\label{ssec:2006/sens/lnox}

\section{Conclusions}\label{sec:2006/conslusion}