\ifpdf
    \graphicspath{{Chapter_2006/figures/PNG/}{Chapter_2006/figures/PDF/}{Chapter_2006/figures/}}
\else
    \graphicspath{{Chapter_2006/figures/EPS/}{Chapter_2006/figures/}}
\fi

\section{Discussion}\label{sec:2006/discussion}

\subsection{Tracer relations}\label{ssec:2006/discuss/tracer}

The relationship between ozone and carbon monoxide (\chem{CO}) has been used to study tropospheric-stratospheric transition in
chemical regimes \citep[e.g.][and references therein]{Pan:2007sw,Hegglin:2009fk}. Similarly, it may be applied to tropospheric data
alone to identify contributions from various pathways \citep[e.g.][]{Zhang:2006zr,Voulgarakis:2011fk,Cristofanelli:2013uq}. Ignoring
\chem{NO_x}, ozone is expected to be strongly positively correlated with \chem{CO} \citep{Chin:1994kx}. However, since the dominating
source for \chem{CO} is surface emission, with a limited lifetime, \chem{CO} decreases rapidly with height. At the same time, significant
contribution of stratospheric ozone and upper tropospheric production cause ozone to increase rapidly with height. Together, \chem{O_3-CO}
correlation over a sufficiently expansive column can expect to observe an "L"-shaped joint distribution.

\figuremacroN{tracer/tracer_correlations}{\chem{CO}-\chem{O_3} model correlation}{\label{fig:2006/o3co}
\textbf{(a)} Anticyclone ($\bar{Z}_{300}>9730\,\unit{m}$) \chem{CO}-\chem{O_3}(adjusted) joint-distribution separately
colored by pressure levels. \textbf{(b)} Same as \textbf{a} but outside the anticyclone between 9710\,\unit{m} and
9730\,\unit{m}. Data points south of 25$^\circ$N have been removed to reduce impact of the southern boundary. \vspace{-.2in}}

Here we correlated the August average model ozone and \chem{CO} from the surface up to 100\,\unit{hPa} (Fig.~\ref{fig:2006/o3co}.
Data points are first separated by within and outside of the anticyclone, defined by the boundary $\bar{Z}_{300}$ as previously
defined in Section~\ref{ssec:2006/gen/ozone}. Then, each panel is further classified by pressure level split at 300 and 700\,\unit{hPa}
to represent the lower, middle, and upper troposphere. To account for model biases, ozone VMRs between 100--300\,\unit{hPa}
have been adjusted by the differences in mean and variance against TES computed in Section~\ref{ssec:2006/gen/ozone}. No
adjustment is performed at the other pressure levels due to insignificant bias compared to observations. While the value shifted,
the shape and features of the joint-distribution remain consistent before and after the adjustment. The labeled features annotated
in the figure are discussed below.

Feature A indicates a type of air mass exist only outside the anticyclone. This unique feature corresponds to subtropical Atlantic
influx from the southern boundary. Typically, it has higher \chem{CO} but the ozone VMRs at 300\,\unit{hPa} from lower latitudes are
lower in general. These air masses intermittently contribute to the anticyclone, similar to the air masses mixed across the jet stream
from the north (see Sect.~\ref{ssec:2006/gen/ozone} \& \ref{ssec:2006/gen/co}). Feature B indicates the upper tropospheric branch
of the \chem{O_3}-\chem{CO} correlation. It is evident from the labels, which are placed at the same locations in both panels, that
\chem{CO} is more effective in raising the ozone level within the upper troposphere above the baseline, indicated by the dashed
line (feature C). Similar to the alphabetic labels, the baseline is also identical in both panels. Thus, it can be seen that the background
\chem{O_3}-\chem{CO} correlations are consistent both inside and outside the anticyclone. Finally, feature D denotes the mid-to-lower
tropospheric branch, which includes the boundary layer. Similar to B, D is chemically different across the anticyclone. Since circulation
is not a factor at low altitudes, the other plausible explanation is the abundant sources of anthropogenic and biogenic emissions of both
VOCs and \chem{NO_x}, which subsequently enhance in-situ ozone production within this layer and transported upwards. Finally,
The low-\chem{CO}, low-\chem{O_3}  value at the interface of lower and middle tropospheres is about 10\,\unit{ppbv} higher within
the anticyclone compared to the $\sim38$\,\unit{ppbv} value outside of the region despite negligible differences in their corresponding
\chem{CO} level.

The analysis above shows that the condition within the anticyclone enables higher ozone values throughout the entire troposphere at
a wide range of VMRs for \chem{CO} (features B and C). Since emission dominates transport in the lower
troposphere, the appropriate hypothesis for the enhancement at C is the elevated level of \chem{NO_x} from either co-located
emissions or subsidence of {\lnox} within the region. However, more information is required to properly diagnose feature B, which
can be either an indication of enhanced \chem{O_3} chemical production at the 80\,\unit{ppbv} \chem{CO} or enhanced \chem{CO}
at levels with \chem{O_3} through convective detrainments.

\figuremacroW{tracer/tracer_maps}{Passive tracers}{\label{fig:2006/tracers}
First row ({\bf a,b}) shows the August average of lateral boundary decaying tracer at 300\,\unit{hPa} on the left and along
33$^\circ$N on the right. Values are given in percentage. Second row ({\bf c,d}) shows the surface layer deacying tracer.
Third row ({\bf e,f}) shows the stratospheric decaying tracer.
Last row ({\bf g,h}) shows the lightning decaying tracer given in \unit{ppbv}.}{0.9}

Figure~\ref{fig:2006/tracers} shows four of the tracers released from the lateral boundaries, the boundary layer, the stratosphere, and
lightning sources. All tracers shown are the decaying type with a lifetime of 1 day (1-d). The impact of the NAM circulation is clearly shown
in the spatial distribution of the lateral boundary (BC) tracer (Fig.~\ref{fig:2006/tracers}{\bf a}). Within the anticyclone region ($\bar{Z}_{300}
>9730\,\unit{m}$), air mass retains no more than an average of 3\% of the characteristics from any of the boundaries. Since these
tracers are dynamically-driven, it can be asserted that the impact of boundary condition on the principle results will be minor.

Similarly, lightning LT) tracer also shows substantial localized enhancement within the anticyclone region (Fig.~\ref{fig:2006/tracers}{\bf g}).
The observed eastward shift from the 9730\,\unit{m} contour can be explained by the overall higher lightning flash rate in the southeastern 
United States. Accounting for the one order of magnitude deviation from observed CG flash rate, the maximum accumulated lightning
1-d tracer is 0.5\,\unit{ppbv}. Despite a fixed maximum emission altitude of 6--8\,\unit{km} ($\sim400\,\unit{hPa}$) according to
\citet{Ott:2010lo}, vertical distribution along the 33$^\circ$N cross-section shows that the tracer has values $>3\,\unit{ppbv}$ (0.3
after account for overprediction in flash rate) within a wide range of pressure levels between 120--500\,\unit{hPa}
(Fig.~\ref{fig:2006/tracers}{\bf h}). Finally, the abundance of lightning tracers in the stratosphere is likely the direct result of using a
fixed altitude-dependent emission profile, which includes more than 10\% (35\,\unit{moles\,flash^{-1}}) above 12\,\unit{km} for
subtropical latitudes.

However, the spatial distribution of boundary layer (BL) tracer shows little correlation with the shape or positioning of the anticyclone. Instead,
the features of the tracer at 300\,\unit{hPa} are dominated by orographic influence of the Rocky Mountains (Fig.~\ref{fig:2006/tracers}{\bf c}).
Similarly, stratospheric (ST) tracer also shows very minimal impact from the NAM circulation except for the influx of stratospheric air eddy
shedding across the jet stream in the outflow region along the eastern seaboard.

\figuremacroW{tracer/age_maps}{Passive tracer age}{\label{fig:2006/tracer_age}
Same as Figure~\ref{fig:2006/tracers} except for tracer equivalent age. Darker colored regions represent young tracers, while lighter
colored/white regions represent aged tracers}{0.9}

The tracer equivalent age, calculated according to Equation~\ref{eqn:tracer-age}, is mapped in Figure~\ref{fig:2006/tracer_age}.
Similar to the decaying tracer distributions, BC and LT tracers are the only ones showing some correlations with the anticyclone boundary.
The range of age for BC tracer within the anticyclone is 2.6--6.8 days while the range outside the anticyclone between 9710\,\unit{m} and
9730\,\unit{m} is 1.5--4.8 days. On the contrary, LT tracer age ranges from 10.9--21.8\,\unit{hrs} within the anticyclone and 13.1--22.5\,\unit{hrs}
just outside. The primary contribution to the LT distribution is that the anticyclone encompasses areas with high lightning flash frequencies,
and thus younger tracers are expected only on the eastern $\sim2/3$ of the anticyclone as well as towards the outflow region along
the East Coast. On the west end of the cross-section (Fig.~\ref{fig:2006/tracer_age}{\bf h}), the presence of high altitude LT tracer
(Fig.~\ref{fig:2006/tracer}{\bf h}) is shown to be disconnected from the high LT tracer core within the anticyclone

\figuremacroN{tracer/passive_correlations}{BL-ST tracer correlation}{\label{fig:2006/blst}
Same as Fig.~\ref{fig:2006/o3co} but for BL and ST decaying tracers. \vspace{-.2in}}

Individually, BL and ST are dominated by orographic or large scale features. By correlating the two tracers, differences emerge.
Figure~\ref{fig:2006/blst} shows the joint distribution between the two decaying tracers. At feature A, or absence thereof within the
anticyclone, low ST-low BL data points exist only in the mid-troposphere (300--700\,\unit{hPa}) outside the anticyclone. Similarly,
feature C (moderate ST/BL) occurs outside the anticyclone only. These two features point to higher heterogeneity of air mass sourcing
outside the anticyclone than within the anticyclone between 700 and slightly above 300\,\unit{hPa}. Such result can be attributed to
the geographic diversity outside the anticyclone, primarily the presence of marine surfaces. Finally, both regions show very little
differences close to the tropopause, where ST$\sim1$.

The result in this section partially contradicts that of \citet{Li:2005ss} and \citet{Cooper:2007cr}, in particular, the conjecture that the anticyclone
enables higher ozone production through recirculation and entrapment of detrained BL air rich in ozone chemical precursors. The result here
shows that the vertical colocation of geographical homogeneity as well as meteorological factors (e.g. high lightning intensity) are the key factors
driving the ozone enhancement. On the other hand, the lack of model lateral boundary influence, as indicative of the BC tracer low within the
anticyclone, points to low externality as a secondary factor as opposed to internal conditioning. The only\footnote{Despite the lack of aircraft
emission in the simulation used in this study, it can be argued that the influence of aircraft emission on upper tropospheric ozone production
may also be enhanced by the transport pattern due to the emission altitude and $x$-$y$ spatial distribution.}exception to the irrelevance of
internal condition is the influence from lightning emissions, which shows correlation with the transport pattern of the anticyclone.

\subsection{Tendency diagnostics}\label{ssec:2006/discuss/tendency}

The changes of the mixing ratio of a chemical species at a specific time and location, or tendency, can be expressed as
the sum of following components: chemistry, convection, vertical mixing, horizontal advection, vertical advection, emission,
and other loss rates (Eq.~\ref{eqn:tendency}). In the upper troposphere, except for \chem{NO}, both emissions and other
uncharacterized loss rates should be zero, and thus the total tendency can be expressed as the first five processes. It is thus
possible to infer changes in both chemical and dynamical properties and relations by examining these tendency components.
Tendency diagnostics of these 5 components are available for \chem{O_3}, \chem{CO}, \chem{NO}, \chem{NO_2}, \chem{HO},
\chem{HO_2}, \chem{TOL}, and \chem{HC5}.\footnote{TOL$=$Toluene; HC5$=$Bulk species for alkane with moderate \chem{HO}
rate constants. These replace isoprene and \chem{HNO_3} in the default implementation.}

\figuremacroW{tendency/timeseries.png}{Time series for upper tropospheric tendency diagnostics}{\label{fig:2006/tendency_timeseries}
Time series at ({\bf a}) Trinidata Head, ({\bf b}) Houston, and ({\bf c}) Beltsville showing the min-mean-max ozone time series (black)
between 270--330\,\unit{hPa} overlaid with IONS-06 ozonesonde measurements represented as red vertical lines for the range and solid
$\diamondsuit$ for the mean value. Lightning \chem{NO_x} 1-d tracer (cyan) and ozone chemical (red), convective (blue), and vertical
mixing (purple) tendency components are given according to the blue axis labels on the right in units of \unit{ppbv}. Boundary layer (BL, blue)
and stratospheric (ST, red) are also given in \unit{\%}.}{0.9}

To characterize the air mass of the continental inflow, the anticyclone, and the continental outflow, three locations are selected to be
examined further: Trinidad Head, California (40.80$^\circ$N, 124.15$^\circ$W), Houston, Texas (29.72$^\circ$N, 95.30$^\circ$W),
and Beltsville, Maryland (39.04$^\circ$N,76.52$^\circ$W). Figure~\ref{fig:2006/tendency_timeseries} shows the time series of the
ozone VMRs at these locations as well as the associated chemical, convective, and mixing tendencies. Values shown are mean for
all data points between 270--330\,\unit{hPa}, with min/max also indicated for ozone VMR. After accounting for the three tendency
included in the figure, the remaining variability in the total VMR can be explained fully by horizontal and advective tendencies.
To indicate model bias, IONS-06 ozonesonde measurements, described in Section~\ref{ssec:2006/gen/ozone}, are also included.
Boundary layer (BL), stratospheric (ST) and lightning (\lnox) decaying tracers, described in Section~\ref{ssec:2006/discuss/tracer}
are included to provide context of the sources for dynamical and chemical variabilities.

The primary source for temporal variabilities at all three locations are advection. In particular, Trinidad Head periodically receives
stratospheric ozone from the jet stream, which is responsible for the sharp spatiotemporal gradients high VMRs simulated for particular
days (e.g. August 2 and August 7). The largest chemical tendency for ozone is observed at Beltsville with a total of 110 \,\unit{ppbv}
over the entire August, followed by Houston with 69\,\unit{ppbv}. Despite a higher total chemical tendency in the upper troposphere above
Beltsville, as formalized by Equation~\ref{eqn:tendency-accul}, Figure~\ref{fig:2006/tendency_timeseries} shows that the upper air
in the southern United States is more chemically active, with a min/max 3-hourly ozone chemical tendencies of -12.8/17.1\,\unit{ppbv},
compared to -2.94/3.90\,\unit{ppbv} at Beltsville.\footnote{One might observe that the ratios at both locations are identical despite the
difference in magnitude. This may be attributed to the fact that Beltsville, being located in the outflow region, inherited a similar chemical
composition as that within the anticyclone.} On the contrary, the inflow region, represented by Trinidad Head, observed $\pm0.37$\,\unit{ppbv}
between all 3-hourly outputs.

%31 sondes found for trinidad
%Reading WRF files
%      192.169      51.4095
%chem     0.352303
%conv     -1.95329
%vmix      8.09430
%abs chem      12.8248
%min/max chem    -0.335217     0.370461
%19 sondes found for houston
%Reading WRF files
%      116.633      66.5038
%chem      68.9099
%conv     -13.4116
%vmix      3.62007
%abs chem      414.107
%min/max chem     -12.7622      17.0969
%10 sondes found for beltsville
%Reading WRF files
%      98.1389      102.416
%chem      110.840
%conv     -5.09015
%vmix      3.52898
%abs chem      175.398
%min/max chem     -2.93698      3.89780
On the other hand, Trinidad Head observed less than 0.4\,\unit{ppbv} over the same period. This is likely due to the lack of precursors,
both VOCs and \chem{NO_x}, within the mid-latitude inflow despite the high ozone. On the contrary, both convective and mixing tendency
components are negligible with -13.4\,\unit{ppbv} convective and 3.6\,\unit{ppbv} mixing at Houston, -5.1\,\unit{ppbv} convective and
3.5\,\unit{mixing} at Beltsville, and -2.0\,\unit{ppbv} convective and 8.1\,\unit{ppbv} mixing at Trinidad Head. The low negative value can
be attributed to infrequent convective activity (e.g. Trinidad Head) or a small gradient between the entrainment level and $\sim300\,\unit{hPa}$
at Houston where convective activity is frequent. Vertical mixing tendency is also similarly dependent on vertical gradient as well as
transport efficiency, which is eddy transport and dry deposition in this case. At Trinidad head, where the tropopause is frequently
lowered due to the jet stream, the model more than twice the vertical mixing tendency compared to the other locations as a result of
the sharpened ozone VMR gradient closer to the tropopause.

\section{Sensitivity study}\label{sec:2006/sens}
\subsection{Anthropogenic emission}\label{ssec:2006/sens/anthrop}
\subsection{Biogenic emission}\label{ssec:2006/sens/bio}
\subsection{Lightning emission}\label{ssec:2006/sens/lnox}

\section{Conclusions}\label{sec:2006/conslusion}