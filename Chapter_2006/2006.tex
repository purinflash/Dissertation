\chapter{2006 North American Monsoon Case Study} \label{ch:2006}

\ifpdf
    \graphicspath{{Chapter_2006/figures/PNG/}{Chapter_2006/figures/PDF/}{Chapter_2006/figures/}}
\else
    \graphicspath{{Chapter_2006/figures/EPS/}{Chapter_2006/figures/}}
\fi

The recurring upper tropospheric ozone enhancements during North American Monsoon (NAM) seasons have been extensively studied by \citet{Li:2005ss}, \citet{Cooper:2006dq,Cooper:2007cr,Cooper:2009nx}, and \citet{Barth:2012qf}. Other studies have also highlighted the relevance of this feature \citep[e.g.][]{Hudman:2007fu,Choi:2009bh,Jourdain:2010tw}. While previous studies produced results that has been used to infer climatological impact of the upper tropospheric ozone enhancement, a closer look at model outputs reveals substantial variabilities. To further the understanding of a chemistry model's capability in capturing the necessary details, this study intends to perform simulations using WRF-Chem \citet{Grell:2005fv} over July and August of 2006. The period covered by this study is chosen to capture substantial anticyclonic recirculation in the upper-air, which has been shown to enhance ozone production \citet{Cooper:2007cr}.

The outputs of the base case simulation is compared to several independent data sets to identify any biases or deficiencies in the model that have implications on the modeled ozone budget. In addition, sensitivity simulations are also performed to evaluate the relative sensitivity of the simulated ozone distributions to emission scenarios of anthropogenic, biogenic, and lightning sources. The results from these simulations are used to bound the range of uncertainty as well as to address the impact from potential changes in emission scenarios in the future.

The first section of this chapter describes the methods used in this study (\sect{2006/method}). Then, a general description of the model outputs and corresponding validation against various empirical data sets are given in Section \ref{sec:2006/general}. Once the model's performance has been identified, Section \ref{sec:2006/discussion} describes the core results derived from the base case simulation. Finally, this chapter is concluded with sets of sensitivity tests for determining ozone's variability with respect to anthropogenic emissions (\ssect{2006/sens/anthrop}), biogenic emissions (\ssect{2006/sens/bio}), and lightning-generated \chem{NO_x} (\lnox) emissions (\ssect{2006/sens/lnox}).

\section{Model description}\label{sec:2006/method}

\figuremacroN{domain}{2006 case study model domain}{\label{fig:2006/domain}WRF-Chem model domain (red). Marked inner region (blue) is the primary area used for various analysis in Section \ref{sec:2006/discussion}.\vspace{-.2in}}

Simulations are formed with Weather Research and Forecasting model \citep{Skamarock:2008xx} with Chemistry \citep[WRF-Chem;][]{Grell:2005fv} version 3.4.1 over July and August, 2006. The model is configured with a horizontal grid spacing of 36\,\unit{km} with a lambert conformal projection centered over the contiguous United States (CONUS) as shown in Figure \ref{fig:2006/domain}. Vertical levels are gridded into 51 eta levels with variable heights up to a fixed top at 10\,\unit{hPa}.

Meteorology is initialized and assimilated with data from National Center for Environmental Prediction (NCEP) Global Forecasting System (GFS) final (FNL) gridded analysis at 6-hr intervals (00, 06, 12, 18\,\unit{UTC}). Nudging is performed for temperature and water vapor above the planetary boundary layer (PBL). Nudging for horizontal winds is performed above model level 10. Advection of moistures, scalars, chemicals, and tracers are performed with positive-definite and monotonic limiters described in \citet{Skamarock:2006wm}.

The rest of this chapter describes specific model settings in details. The full namelist template for the base case simulation is also provided as Appendix \ref{apdx:namelist}.

\subsection{Model options}\label{ssec:2006/method/settings}

	Convective parameterization is facilitated with the Grell-3 (G-3) scheme, a modified version of the \citet{Grell:2002bs} ensemble scheme. It uses a $144(=3\times3\times16)$ dynamic control and static control/feedback closures with varying parameters. In addition, shallow convection is also enabled to simulate less severe convective activities. For microphysics, the \citet{Thompson:2008vn} scheme is used, which allows a generalized gamma distribution for graupel.

	Shortwave radiative transfer uses the Goaddard two-stream method described by \citet{Chou:1998kx} with 11 bands using a $\delta$-Eddington approximation for scattering and transmission. It also takes into account third-order effects from molecular absoprtions. Longwave radiative transfer uses the Rapid Radiative Transfer Model \citep[RRTM;][]{Mlawer:1997vn}. The NOAH Land Surface Model \citep{Chen:2001ys} is used for the land surface and the Yonsei University (YSU) scheme  \citep{Hong:2006fk} is used for boundary layer physics.

\subsection{Chemistry and emissions}\label{ssec:2006/method/chem}

	The chemical mechanism used in this study is the Regional Acid Deposition Model version 2 \citep[RADM2;][]{Stockwell:1990ez} compiled with a ``WRF-conformed'' version of the Kinetic Preprocessor \citep[KPP;][]{Sandu:2006jl}. KPP uses a Rosenbrock approach, a iterative predictor-corrector method described by \citep{Hairer:1993zr}, to solve the highly nonlinear stiff system of differential equations consisting of 63 species, 21 photolysis, and 136 gas-phase reactions. No aerosol is included in any of the simulations. Photolysis is calculated using the fast Tropospheric Ultraviolet-VIsible \citep[FTUV;][]{Tie:2003ve} scheme. However, several implementation errors, which have largely been identified and patched after substantial efforts (see Appendix \ref{a-sec:bug/ftuv}).

	Chemical initial and boundary conditions are obtained from the Model for OZone and Related chemical Tracers (MOZART-4) global chemistry model \citep{Emmons:2010fk}, which includes 85 gas-phase species, 39 photolysis, and 157 gas-phase reactions. Therefore, species mapping is performed especially for some of the bulk species (e.g. \chem{OLT}\,\texttt{:=}\,\chem{C_3H_6}+\chem{MVK}+0.5\chem{BIGENE}\footnote{\chem{OLT}=terminal alkenes in RADM2; \chem{MVK}=methyl vinyl ketone, \chem{BIGENE}=\chem{C_{4+}H_8} in MOZART-4}). {\lnox} emission uses a modified \citet{Price:1992wb} method described and evaluated in detail in Chapter~\ref{ch:lightning} as well as \citet{Wong:2013vn}. Base case simulation uses an emission factor of 350\,\unit{moles\,\chem{NO}/flash} and vertical distributions described in \citet{Ott:2010lo}. Further improvements have also been implemented for WRF-Chem version 3.5, to be released in mid-2013 (Appendix~\ref{apdx:lnox-doc}).
	
	% yellowstone:/glade/p/work/johnwong/final_inputs/emiss
	\figuremacroW{anthrop}{EPA NEI05 \chem{NO} emission}{\label{fig:2006/anthrop} (a) NEI05  \chem{NO} emission during weekdays at 12Z after mapping to WRF grid. (b)  \chem{NO} emission at Houston,TX between 12Z--23Z as an example for demonstrating the differences between emissions for different days of the week.\vspace{-.2in}}{1.0}
	
	\subsubsection{Anthropogenic emissions}\label{sssec:2006/method/emiss/anthrop}

	The 2005 National Emission Inventory (NEI05) from the Environmental Protection Agency (EPA) is used to define the anthropogenic emission over CONUS. Three sets of data are separately available for weekdays (Monday -- Friday), Saturdays, and Sundays respectively and are swapped in accordingly at 06Z on each simulated day. Figure \ref{fig:2006/anthrop} uses \chem{NO} as an example to demonstrate a typical national emission distribution and differences between days of the week. The utility ``emiss\_v03'' provided by NOAA/ESRL is used with the lambert conformal mapping function rewritten to be consistent with that of WRF's Preprocessing System (WPS) version 3.4.1. The NEI05 source data includes \chem{CO}, \chem{NO_x}, \chem{SO_2}, \chem{NH_3}, 40 hydrocarbon aggregated species, 5 PM2.5 groups, and PM10. Each species is then mapped into its corresponding counterpart in RADM2. Table \ref{table:2006/emiss_v03} shows a few examples of how the species mapping is done. Since the case study is performed without aerosols, the aerosol products produced by the utility are not used.
	
	\begin{table}[htb]
	\caption[Examples of NEI-RADM2 speciation mapping]{Examples of how species are mapped from EPA NEI to RADM2 for anthropogenic emission in the case study.}
	\begin{center}
	\begin{tabular}{cccc}\tophline 
		{\bf NEI species} & {\bf RADM2 species} & {\bf Weight factor} & {\bf Description}  \\ \middlehline
		CO & co & 1.00 & Carbon monoxide\\  
		NOX & no & 1.00 & nitrogen oxides \\  
		SO2 & so2 & 1.00 & Sulfur dioxide \\ 
		HC02 & eth & 1.00 & Ethane kOH $<$ 500 \unit{ppm^{-1}~min^{-1}}\\  
		HC03 & hc3 & 1.00 & Alkane 500 $<$ kOH $<$ 2500 ${}^a$\\  
		HC04 & hc3 & 1.11 & Alkane 2500 $<$ kOH $<$ 5000 ${}^b$\\  
		HC05 & hc5 & 0.97 & Alkane 5000 $<$ kOH $<$ 1e4 ${}^c$\\  
		HC06 & hc8 & 1.00 & Alkane kOH $>$ 1e4 \\  
		HC07 & ol2 & 1.00 & Ethylene\\  
		$\cdots$ &  &  & \\  
		HC18 & ket & 0.33 & Acetone\\  
		HC19 & ket & 1.61 & Methylethyl ketone\\  
		HC20 & ket & 1.61 & PRD2 SAPRAC species (ketone)\\  
		$\cdots$ &  &  & \\  
		PM01 & pm25i & 0.20 & Unspeciated nuclei mode PM2.5\\  
		PM01 & pm25j & 0.80 & Unspeciated accumulation mode PM2.5\\  
		$\cdots$ &  &  & \\ \bottomhline 
	\end{tabular} \label{table:2006/emiss_v03}
	\end{center}{
	\footnotesize
	${}^a$ Excluding \chem{C_3H_8}, \chem{C_2H_2}, organic acids.
	\\ ${}^b$ Excluding butanes.
	\\ ${}^c$ Excluding pentanes.
	}
	\end{table}

	\subsubsection{Biogenic emissions}\label{sssec:2006/method/emiss/bio}

	Unlike anthropogenic emission, Model for Emissions of Gases and Aerosols from Nature \citep[MEGAN2;][]{Guenther:2006kl} is used to drive a biogenic emission parameterization within WRF-Chem. MEGAN2 uses a combination of climatological leaf area index (LAI), vegetation speciation, model temperature, and model solar radiation to compute BVOC emissions consistent with model meteorological states, with isoprene being the primary species of interest (see \ssect{intro/ozone/voc}).
	
	The general formulation of BVOC emissions is as follow \citep[after][]{Guenther:2006kl}:
	\begin{equation}\label{eqn:MEGAN-E}
		\mathrm{Emission} =\varepsilon\cdot\gamma\cdot\rho
	\end{equation}
	where $\varepsilon$ is the emission of a compound under standard climatological conditions and is spatially varying,  $\gamma$ is the emission activity factor that accounts for deviation from climatology, and $\rho$ is a factor that accounts for production and loss within the plant canopy, which is assigned as 1.0 in WRF-Chem.
	
	The emission activity factor $\gamma$ is the product of multiple factors accounting for the canopy environment (CE), age of the vegetations, and soil moisture (SM):
	\begin{equation}\label{eqn:MEGAN-gamma}
		\gamma = \underbrace{\gamma_{LAI}~\cdot~\gamma_P~\cdot~\gamma_T}_{\gamma_{CE}}~\cdot~\gamma_{age}~\cdot~\gamma_{SM}
	\end{equation}
	Like the production/loss factor, $\gamma_{SM}$ is set to 1.0 in WRF-Chem. The age factor $\gamma_{age}$ classifies foliage into 4 types: new, growing, mature, and old, of which the respective weighing is computed using the changes in leaf area indices.
	
	Instead of a detailed canopy model, $\gamma_{CE}$ is represented as three factors controlled by leaf area, photosynthetic photon flux density (PPFD), and temperature. Such approach is referred to as the parameterized canopy environment emission activity (PCEEA) algorithm, which produces annual global isoprene emissions within $\sim 5\%$ of standard canopy model but may exceed 25\% for specific times and locations \citep{Guenther:2006kl}. The temperature factor is highly nonlinear and depends on current and past averages of the leaf temperature:
	\begin{eqnarray}
		\gamma_T &=& E_{opt}\frac{C_{T2}\exp(C_{T1}\cdot x)}{C_{T2}-C_{T1}(1-\exp(C_{T2}\cdot x))} \label{eqn:MEGAN-gammaT}  \\
		x &=& \left[1/T_{opt}-1/T\right]\large/0.00831 \\
		T_{opt} &=& 313 + 0.6 \cdot (T_{240}-297) \\
		E_{opt} &=& 2.034 \exp[0.05(T_{24}-297)]\exp[0.05\cdot(T_{240}-297)]
	\end{eqnarray}
	where $T$ is the leaf temperature in Kelvin, $C_{T1}=95$ and $C_{T2}=230$ are empirical coefficients, $T_{24}$ and $T_{240}$ are average leaf temperature over the past 24 and 240 hours respectively. On the other hand, the LAI factor is sublinear:
	\begin{equation}\label{eqn:MEGAN-gammaLAI}
		\gamma_{\mathrm{LAI}} = \frac{0.49~\mathrm{LAI}}{\sqrt{1 + 0.2~(\mathrm{LAI})^2}}
	\end{equation}
	Therefore, LAI factor is the most sensitive to changes at low LAI  and evaluates to 1.0 at LAI=5.0.

\subsection{Tendency diagnostics and Passive tracers}\label{ssec:2006/method/diagnostics}

	Budgeting the tendencies, or attributing changes, of a chemical or aerosol species has been the primary goal of many studies. For air quality studies using limited ground stations, such task has been performed using decomposition methods such as Principal Component Analysis (PCA) \citep[][and references therein]{Langford:2009mb}. However, PCA is unable to properly decompose correlated tendencies and thus the apportionment may be incorrect.
	
	Presented here is a simple method implemented into WRF-Chem since version 3.3. It leverages the deterministic nature of a numerical weather-chemistry model and extracts the relevant tendencies prior to coupling without loss of information. Conceptualize the tendency calculation of a scalar $S$ from time step $t$ to $t+1$ as follow:
	\begin{equation}\label{eqn:tendency}
		S^{(t+1)}-S^{(t)} = (\Delta_{chem}+\Delta_{conv}+\Delta_{vmix}+\mathbf{v}\cdot\nabla + w\delta_z)S^{(t)}\delta t + E_S^{(t)} + LS^{(t)}
	\end{equation}
	where $\Delta$'s are the time-stepping operators for chemistry, convection, and vertical mixing, $\mathbf{v}$ is the horizontal wind vector, and $w$ is the vertical wind. $E_S^{(t)}$ and $LS^{(t)}$ are the emission and loss terms. Then, we may accumulate at every single time step for a specific process $p$ to obtain the total tendency $T$ for a scalar $S$ due to the said process up to time $t$ from initialization :
	\begin{equation}\label{eqn:tendency-accul}
		T_{p}^{(t)} = \sum_{\tau=0}^t\Delta_{p} S^{(\tau)}\delta\tau
	\end{equation}
	While Equation \ref{eqn:tendency-accul} appears first-order, the tendencies obtained from each time step is calculated with the accuracy of the respective method used. For example, advection in the simulations employed in this study use a method that has a lower order term and a monotonic limiter described in \citet{Skamarock:2006wm}. To compute the two advective tendencies, both terms are included into the accumulated diagnostic outputs. Therefore,  if $\langle E_S^{(\tau)}+LS^{(\tau)}\rangle_{\tau<t}=0$, the following is expected to hold with the processes $\mathcal{P}$ from Equation \ref{eqn:tendency}:
	\begin{equation}\label{eqn:tendency-good}
		\sum_{p\in\mathcal{P}}T_{p}^{(t)} \equiv S^{(t)}-S^{(0)}
	\end{equation}
	
	Currently, the default implementation includes 5 processes: chemistry, convection, vertical mixing, horizontal advection, and vertical advection. $T_{p}^{(t)}$ for 8 species are computed for this study: \chem{O_3}, \chem{CO}, \chem{NO}, \chem{NO_2}, \chem{HO}, \chem{HO_2}, \chem{CH5}\footnote{Bulk species for alkane with moderate \chem{HO} rate constants.}, and \chem{MGLY}\footnote{Methyl glyoxal.}. These diagnostic outputs are used extensively throughout this study.
	
	Another subject of frequent interest is the source and age of an air mass. It is often achieved by including online point-sourced tracers or computing Lagrangian trajectories offline with tools such as FLEXPART \citep{Stohl:2005vn}. Of interest in this study are air masses from the boundary layer, stratosphere, lateral domain boundaries, and those affected by {\lnox} emission. Values of the lateral boundaries, boundary layer, and stratospheric tracers are held at 1.0 at their sources and allowed to be freely transported through advection, convection\footnote{By sharing convective transport method with chemical species, an implementation error was introduced. See \mbox{Appendix}~\ref{a-sec:bug/ctrans} for more details.}, and vertical mixing. Similarly, lightning tracers are also emitted at the same rate and locations as those determined by the {\lnox} parameterization described in Chapter \ref{ch:lightning}, in both passive and decaying form. All decaying twins of these tracers decay with an $e$-folding time of one day. An equivalent mean tracer age within a volume $V$ may be defined as
	\begin{equation}\label{eqn:tracer-age}
		a_e = (\mbox {1 day})\times\ln\left(\int_V{\phi_0}dV/\int_V\phi_1dV\right)
	\end{equation}
	where $\phi_0$ is the passive non-decaying tracer, and $\phi_1$ is the decaying twin. Such definition is chosen over $\int_V\phi_0/\phi_1dV$ to avoid cases where $\phi_0\sim\phi_1\sim0$.

%
% Starting in this section, paragraphs will be separated into adjacent lines to allow finer version control by git
%
\section{General results and validation}\label{sec:2006/general}

This section describes various aspects of the simulation results and compares them against
corresponding observational data sets. The primary goal of this section is to provide context
and bounds for uncertainty for further discussions in Section \ref{sec:2006/discussion}.

\subsection{Meteorology}\label{ssec:2006/gen/met}

The influence of meteorology on chemistry is of significant importance for the formation of the NAM
ozone enhancement \citep{Li:2005ss,Cooper:2007cr,Barth:2012qf}. In particular, convection has
been shown to detrain boundary layer air, rich in ozone precursors, into the upper troposphere and
thus perturbing ozone distributions \citep{Dickerson:1987hc,Kar:2004jl,Weinstock:2007yj}. Moreover,
intense convective activities also generate thunderstorms responsible for {\lnox} emissions, which
further amplifies the impact of convective activities on tropospheric ozone chemistry. On the synoptic
scale, anticyclonic circulation in the upper troposphere has been attributed to retain detrained ozone
precursors, allowing prolonged ozone production over the southern United States
\citep{Li:2005ss,Cooper:2007cr}.

\figuremacroW{qv_fdda-3hr}{WRF and NWS July and August precipitation}{
\label{fig:2006/precipmap}
\textbf{(a)} WRF-simulated total precipitation,
\textbf{(b)} WRF-simulated convective precipitation, and 
\textbf{(c)} total NWS AHPS precipitation in \unit{mm/day}.
\textbf{(d)} Parameterized fraction (\%) of model simulated precipitation.}{.9}

The primary meteorological feature of interest in this study is convection, which is validated via precipitation
as the proxy. While precipitation is not a good measurement of the immediate convective strength, the
accuracy and availability of information from the National Weather Service (NWS) Advanced Hydrological
Prediction Service (AHPS) allows continuous validation of model's prediction. The data product used here is the
daily precipitation from NWS AHPS, a national mosaic product using the combined data from 12 River Forecast
Centers (RFCs). Estimation and improvement of observed precipitation from the NWS RFCs are produced by
a Multi-sensor Precipitation Estimator (MPE), which combines data from radar and rain gauges across
the united States. Post-analyses are also performed manually by forecasters at NWS to identify any systematic
errors. The resulting data are gridded onto Hydrologic Rainfall Analysis Project (HRAP) grid with dimensions
{\resolution{4}} every 24 hours ending at 12\,\unit{UTC} each day
(\url{http://www.srh.noaa.gov/abrfc/?n=pcpn\_methods}).

\figuremacroN{precip_ts}{WRF and NWS precipitation time series}{\label{fig:2006/precipts}
\textbf{(a)} WRF and NWS daily area mean precipitation within the inner analysis domain shown in
Figure~\ref{fig:2006/precipmap} during August 2006. \textbf{(b)} Frequency distribution within analysis domain
with bin size 0.2\,\unit{mm/day}.\vspace{-.2in}}

Figure~\ref{fig:2006/precipmap} shows a comparison of the spatial distribution of the simulated precipitation
amount by WRF and the observed precipitation from NWS AHPS during the July and August of 2006 with
Figure~\ref{fig:2006/precipmap}(a) being the total precipitation and (b) being the fraction generated by
convective parameterization. Since NWS AHPS does have coverage over marine regions, for clear comparison,
WRF outputs have been masked with the land mask. From the figure, WRF simulation is shown to produce a
comparable spatial distribution to NWS, with high bias at the Arkansas/Texas border and a low bias over
Tennessee and Kentucky. Another low bias is located in North Carolina east of the Blue Ridge Mountains.
Simulated coastal rainfall north of the Gulf Mexico is also generally lower than observed except for regions
near Houston, TX.

During the 28-day period shown in Figure~\ref{fig:2006/precipmap}(a), WRF predicted 70.74\,\unit{mm} area mean
precipitation within the analysis domain while NWS observed 82.29\,\unit{mm}. This -14\% bias in precipitation
is caused by under-prediction in the frequency of heavy precipitation events above 15\,\unit{mm/day}. A likely
cause is that weak/smaller-scale convective events have been over-predicted while stronger/grid-scale convective
events are under-predicted. An attempt to improve precipitation prediction by reducing the timescale for nudging
water vapor to 1\,\unit{hr} instead of 3\,\unit{hr} has instead shown to increase the negative bias to -61\% instead.
Considering the model bias in the mean area prediction for individual day, 12 out of 28 days has less than
10\% absolute bias in the area mean against NWS and 8 out of 28 days are closer than 2\% absolute bias.

There are several reasons why skill score and threat score calculations, both common and appropriate methods for
model precipitation evaluations, are not used here. The primary reason is that the requirement of this study is not to predict
the exact locations of storms, but rather to predict an overall area-integrated convective strength consistent with that
observed. The second reason is that the differences in resolution. A small sub-gridsized storm are represented as
a single activity within a {\resolution{36}} grid whereas it may only generated precipitation within one {\resolution{4}} grid
in the NWS data set.

\subsection{Ozone}\label{ssec:2006/gen/ozone}
\subsection{Carbon monoxide}\label{ssec:2006/gen/co}
\subsection{Formaldehyde}\label{ssec:2006/gen/form}
\subsection{Nitrogen Oxides}\label{ssec:2006/gen/nox}

\section{Discussion}\label{sec:2006/discussion}

\section{Sensitivity study}\label{sec:2006/sens}
\subsection{Anthropogenic emission}\label{ssec:2006/sens/anthrop}
\subsection{Biogenic emission}\label{ssec:2006/sens/bio}
\subsection{{\lnox} emission}\label{ssec:2006/sens/lnox}

\section{Conclusions}\label{sec:2006/conslusion}