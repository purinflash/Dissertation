\chapter{2006 North American Monsoon Case Study} \label{ch:2006}

\ifpdf
    \graphicspath{{Chapter_2006/figures/PNG/}{Chapter_2006/figures/PDF/}{Chapter_2006/figures/}}
\else
    \graphicspath{{Chapter_2006/figures/EPS/}{Chapter_2006/figures/}}
\fi

The recurring upper tropospheric ozone enhancements during North American Monsoon (NAM) seasons have been extensively studied by \citet{Li:2005ss}, \citet{Cooper:2006dq,Cooper:2007cr,Cooper:2009nx}, and \citet{Barth:2012qf}. Other studies have also recognized the importance of this feature \citep[e.g.][]{Hudman:2007fu,Jourdain:2010tw}. While previous studies produced results that has been used to infer climatological impact of the upper tropospheric ozone enhancement, a closer look at model outputs reveals substantial variabilities. To further the understanding of a chemistry model's capability in capturing the necessary details, this study intends to perform simulations using WRF-Chem \citet{Grell:2005fv} over July and August of 2006. The period covered by this study is chosen to capture substantial anticyclonic recirculation in the upper-air, which has been shown to enhance ozone production \citet{Cooper:2007cr}.

The outputs of the base case simulation is compared to several independent data sets to identify any biases or deficiencies in the model that have implications on the modeled ozone budget. In addition, sensitivity simulations are also performed to evaluate the relative sensitivity of the simulated ozone distributions to emission scenarios of anthropogenic, biogenic, and lightning sources. The results from these simulations are used to bound the range of uncertainty as well as to address the impact from potential changes in emission scenarios in the future.

The first section of this chapter describes the methods used in this study (\sect{2006/method}). Then a general description of the model outputs is given (\sect{2006/general}) to provide context. The model is then validated in detail against multiple data sets in Section \ref{sec:2006/validation}, with which the details of the data sets are also given. Once the model's performance has been identified, Section \ref{sec:2006/discussion} describes the core results derived from the base case simulation. Finally, this chapter is concluded with sets of sensitivity tests for determining ozone's variability with respect to anthropogenic emissions (\ssect{2006/sens/anthrop}), biogenic emissions (\ssect{2006/sens/bio}), and lightning-generated \chem{NO_x} (\lnox) emissions (\ssect{2006/sens/lnox}).

\section{Method}\label{sec:2006/method}

\subsection{Model setup}\label{ssec:2006/method/setup}

\subsection{Passive tracers}\label{ssec:2006/method/tracers}

\subsection{Tendency diagnostics}\label{ssec:2006/method/tendency}

\section{General results}\label{sec:2006/general}

\section{Validation}\label{sec:2006/validation}

\section{Discussion}\label{sec:2006/discussion}

\section{Sensitivity study}\label{sec:2006/sens}
\subsection{Anthropogenic emission}\label{ssec:2006/sens/anthrop}
\subsection{Biogenic emission}\label{ssec:2006/sens/bio}
\subsection{{\lnox} emission}\label{ssec:2006/sens/lnox}

\section{Conclusions}\label{sec:2006/conslusion}