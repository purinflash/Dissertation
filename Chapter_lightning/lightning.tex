\chapter{Lightning parameterization} \label{ch:lightning}

\ifpdf
    \graphicspath{{Chapter_lightning/figures/PNG/}{Chapter_lightning/figures/PDF/}{Chapter_lightning/figures/}}
\else
    \graphicspath{{Chapter_lightning/figures/EPS/}{Chapter_lightning/figures/}}
\fi

Over the last decade, predictions of lightning flash statistics in numerical
weather and climate models have garnered increasing interests. One of the
likely drivers is the advances in online chemistry models, wherein chemistry
is simulated alongside of physics \citep[e.g.,][]{Grell:2005fv}.
Lightning-generated nitrogen oxides (\lnox) are predicted to be very
efficient in accelerating the production of tropospheric ozone, which is
identified as a significant greenhouse gas in the upper troposphere
\citep{Lacis:1990fk,Kiehl:1999uq}. \citet{Cooper:2007cr}, showing that during
the summertime North American Monsoon, lightning can contribute 25--30\,ppbv
of upper tropospheric ozone. \citet{Choi:2009bh} has remarked on the increasing
importance of \chem{LNO_x} in tropospheric ozone production as anthropogenic
sources of \chem{NO_x} are being reduced in the United States. Furthermore,
the inherent nonlinearity between \chem{NO_x} emission and commonly validated
quantities such as radiative balances and ozone concentration makes it
challenging to quantify the skill of a {\lnox} parameterization through
proxy or total \chem{NO_x} measurements. Therefore, it is important to
evaluate existing lightning parameterizations by directly validating flash
rate predictions in order to more accurately interpret results from models
that incorporate {\lnox} emission.


The most commonly used method for parameterizing lightning flash rate is
perhaps that by \citet{Price:1992wb,Price:1993fk,Price:1994fk}. It has been
used by chemistry transport modeling studies such as GEOS-Chem
\citep{Hudman:2007fu}, MOZART-4 \citep{Emmons:2010fk}, and CAM-Chem
\citep{Lamarque:2012fk}. Continental flash rates are related to the
fifth-power of cloud-top height by \citet{Williams:1985fk} and
\citet[][hereafter PR92]{Price:1992wb} through empirical evidences that are
consistent with the theoretical scaling arguments of \citet{Vonnegut:1963aa}.
The partitioning between intracloud and cloud-to-ground flashes, or IC\,:\,CG
ratio, is estimated with a fourth-order polynomial of cold cloud-depth, i.e.,
distance between freezing level and cloud-top, in \citet[][hereafter
PR93]{Price:1993fk}. Finally, the parameterization is generalized for
different grid sizes with an extrapolated ``calibration factor'' in
\citet[][hereafter PR94]{Price:1994fk}.

The goals of this study are to evaluate the cloud-top height based
parameterization (PR92, PR93, and PR94) across the bridging resolutions
between those commonly used by global chemistry models
($\Delta x\sim\mathrm{O}(1^\circ)$) and cloud-resolving models ($\Delta x < 5\,\unit{km}$),
and report on statistics over time periods useful for
studying upper tropospheric chemistry (O(\unit{month}))
\citep{Stevenson:2006fk}. It is, however, not the goal of this study to
invalidate previous studies, but to draw attention to the need for careful
implementation and validation of the use of these parameterizations. Here we
report on experiments using PR92, PR93, and PR94 implemented into the Weather
Research and Forecasting model \citep[WRF;][]{Skamarock:2008xx}, focusing on
results from simulations performed at 36\,\unit{km} and 12\,\unit{km}
grid-spacing. A simulation at 4\,\unit{km} grid spacing for 2\,weeks in July
and August 2006 is also analyzed to demonstrate how PR92 behaves
transitioning from cloud-parameterized to cloud-permitting resolutions and
provide insights on how or whether such transition can be done.

 Similar studies have been performed for global models \citep[e.g.,][]{Tost:2007rw},
 but previous regional-scale modeling studies utilizing PR92 at comparable horizontal
 grid spacings have not provided evaluations of the lightning parameterization. Thus,
 there has been insufficient information to understand the behavior of PR92 in this regime.
 Even though these formulations were derived using near-instantaneous data at a cloud-permitting
 resolution (5\,\unit{km}), past applications often utilize temporally and spatially averaged cloud-top
 height outputs or proxy parameters. While the effects of spatial averaging is addressed by the PR94
 scaling factor, effects of temporally averaging cloud-top heights are rarely addressed and may lead
 to significant underestimation due to the fifth-power sensitivity \citep{Allen:2002fk}. Addressing
 the potential issue of temporal averaging, instantaneous cloud-top heights and updraft velocities
 at each time step are leveraged. Comparisons are then performed for temporal, spatial, and spectral features.

The next section (Sect.~\ref{sec:lightning/method}) outlines the methods used in this
study, which includes the formulation and overview of the parameterization
(Sect.~\ref{ssec:lightning/param}), relevant aspects of the model setup, practical
considerations of implementing PR92 (Sect.~\ref{ssec:lightning/model}), and the data
used for validation (Sect.~\ref{ssec:lightning/data}). Section~\ref{sec:lightning/results}
describes the model results and discusses the implications of various
statistics from validation against observations of precipitation, flash rate,
and IC\,:\,CG ratios. Section~\ref{sec:lightning/resolution} discusses how the
performance of PR92 transitions between different resolutions
(Sect.~\ref{ssec:lightning/gridsens}) and between theoretically similar formulations
(Sect.~\ref{ssec:lightning/formsens}). Finally, Sect.~\ref{sec:lightning/conclusions} provides
a summary of key results and cautionary remarks on specific aspects of the
utilization of PR92, PR93, and PR94.

\section{Methods}\label{sec:lightning/method}
\subsection{Parameterization overview}\label{ssec:lightning/param}

In PR92, a fifth-power relation between continental lightning flash rate
($f_\textrm{c}$) and cloud-top height ($z_{\mathrm{top}}$) is established with
observational data following the theoretical and empirical frameworks of
\citet{Vonnegut:1963aa} and \cite{Williams:1985fk}. Assuming a dipole
structure with two equal but opposite charge volumes and a cloud aspect ratio
of approximately one, it is first formulated, based on scaling arguments of
\citet{Vonnegut:1963aa}, that the flash rate would be proportional to maximum
vertical updraft velocity ($w_{\max}$) and fourth-power of cloud dimension.
Imposing a linear relation between $w_{\max}$ and cloud dimension, the flash
rate relationship can be reduced to fifth power of $z_{\mathrm{top}}$
\citep{Williams:1985fk}. It is empirically fit to radar and flash rate data
from several measurements between 1960--1981 to give the continental
equation \citep{Price:1992wb}:
\begin{equation}\label{eq:f_c(z)}
f_{\mathrm{c}}(z_{\mathrm{top}}) = 3.44\times10^{-5}z_{\mathrm{top}}^{4.9}.
\end{equation}
PR92 also estimated that $w_{\max}=1.49z_{\mathrm{top}}^{1.09}$ for
continental clouds, thus allowing a second formulation based on maximum
convective updraft:
\begin{equation}\label{eq:f_c(w)}
 f_{\mathrm{c}}(w_{\max}) = 5\times10^{-6}w_{\max}^{4.54}.
 \end{equation}
A separate formulation of second-order, instead of fifth-order, is also
derived by \cite{Price:1992wb} for marine clouds, for which updraft velocity
is observed to be significantly slower:
\begin{equation}\label{eq:f_mPR92}
f_{\mathrm{m
(PR92)}}(z_{\mathrm{top}})=6.2\times10^{-4}z_{\mathrm{top}}^{1.73}.
 \end{equation}
Taking into account effects from cloud condensation nuclei,
\cite{Michalon:1999fk} modified the marine equation to fifth-order:
\begin{equation}\label{eq:f_mM99}
 f_{\mathrm{m (M99)}}(z_{\mathrm{top}}) =
 6.57\times10^{-6}z_{\mathrm{top}}^{4.9}.
 \end{equation}
The practical viability of the continental relation was proven by
\citet{Ushio:2001kx} and \cite{Yoshida:2009vn} through several case studies.
However, \cite{Boccippio:2002uq} showed that the marine equations are
formally inconsistent with \citet{Vonnegut:1963aa}, and that the marine
equations cannot be inverted to produce cloud tops within the range of
cloud-top observations.

\citet{Price:1993fk} used the flash data from eleven states in the western
United States, detected by wide-band magnetic direction finders, in
combination with thunderstorm radar and radiosondes data to find a relation
for the IC\,:\,CG ratio ($Z$) from cold-cloud depth ($d$), defined as the
distance from freezing level to cloud-top.
\begin{equation}\label{eq:Z}
Z = 0.021d^4-0.648d^3+7.49d^2-36.54d + 63.09
\end{equation}

In \citet{Price:1994fk}, a ``calibration factor'' ($c$) for the resolution
dependency of PR92 is introduced by regridding 5\,\unit{km} data between 1983
and 1990 from the International Satellite Cloud Climatology Project dataset
\citep[ISCCP;][]{Rossow:1991aa} to different horizontal grid sizes. The
resulting equation is as follows:
      \begin{equation}\label{eq:calib}
            c = 0.97241\exp(0.048203R)
      \end{equation}
where $R$ is the grid area in squared degrees. \citet{Price:1994fk} claim
that there is no dependence of $c$ on latitude, longitude, or season. For the
grid sizes used in this study, the values of $c$ are 0.9774 for
36\,\unit{km}, 0.973 for 12\,\unit{km}, and 0.9725 for 4\,\unit{km}.


\begin{table}[t]
\caption{WRF simulations performed in this study.}
\begin{tabular}{ccccc} \tophline
Case \# & d$x$ (km) & d$t$ (\unit{s}) & Output &
Duration \\
\middlehline
1 & 36 & 90 & hourly & JJA 2006 \\
2 & 36 & 90 & hourly & JJA 2011 \\
3 & 12 & 36 & 3-hourly &Jul 2011 \\
4 & 4 & 12 & hourly &25 Jul--7 Aug 2006 \\
\bottomhline
\end{tabular}
\label{tab:lightning/setup}
\end{table}

\subsection{Model setup and implementation}\label{ssec:lightning/model}

Simulations in this study are performed using the Weather Research and
Forecasting (WRF) model version 3.2.1 \citep{Skamarock:2008xx} over the
contiguous United States (CONUS), including part of Mexico and Canada
(Fig.~\ref{fig:lightning/domain}). The simulations have slightly different model
domains because the simulations were developed and performed for objectives
independent of validating the lightning parameterization. Meteorology is
initialized and continuously nudged (horizontal winds, temperature, water
vapor) with the National Center for Environmental Prediction (NCEP) Global
Forecasting System (GFS) final (FNL) gridded analysis at 6\,h intervals
(00:00~UTC, 06:00~UTC, 12:00~UTC, 18:00~UTC).
Four simulations are performed~(Table~\ref{tab:lightning/setup}),
two at 36\,\unit{km} grid spacing, one at 12\,\unit{km} grid spacing, and one
at 4\,\unit{km} grid spacing. All cases use the same vertical coordinates
with 51 sigma levels up to 10\,\unit{hPa}. The Grell--Devenyi ensemble
convective parameterization \citep{Grell:2002bs} with \citet{Thompson:2008vn}
microphysics is used for the simulations where grid-spacing $\Delta x >
10$\,km, for which a convective parameterization is needed. The
implementation of the GD scheme employed in this study consists of $3 \times
3 \times 16=144$ ensemble members comprising of interactions between
different dynamic control and static control/feedback closures. The maximum
moist static energy (MSE) is then used as input with entrainment to calculate
the level neutral buoyancy (LNB), or cloud top. For further information about
the convective parameterization, readers are encouraged to refer to
\citet{Grell:1993dz}.

\begin{figure}[t]
      \includegraphics[width=8.3cm]{gmd-2012-94-f01.pdf}
      \caption{Non-nested domains for WRF simulations and region for analysis.}
      \label{fig:lightning/domain}
\end{figure}

Since the simulations were designed independently, some physics options used
are not consistent. The planetary \mbox{boundary} layer (PBL) parameterization is
handled by the Yonsei University scheme \citep{Hong:2006fk} at 36\,\unit{km}
and Mellon--Yamada--Janjic (MYJ) scheme \citep{Janjic:1994fk} at 12\,\unit{km}
and 4\,\unit{km}. At 36\,\unit{km}, the surface layer physics option used is
based on Monin--Obukhov similarity theory. The surface layer option used at
12\,\unit{km} and 4\,\unit{km} is also based on Monin-Obukhov theory but
includes Zilitinkevich thermal roughness length.

While theoretically the scaling argument of \citet{Vonnegut:1963aa} does not
distinguish between definitions of cloud-top height, the data used to derive
the PR92 relation are radar reflectivity cloud-top heights at a certain
reflectivity threshold. In the WRF implementation of Grell--Devenyi convective
parameterization, the level of neutral buoyancy (LNB) is computed, with
convective entrainment and detrainment accounted for within the calculation
of cloud moist static energy, and readily available as a proxy for sub-grid
cloud-top height. Thus, instead of 20\,\unit{dBZ} reflectivity cloud top,
$z_{\mathrm{top}}$ is approximated by reducing LNB by 2\,\unit{km}, which
will be shown to produce results within the range of the observed values. The
choice of 2\,\unit{km} reduction is made independent of, but supported by, a
recent study comparing different definitions of LNB and found the traditional
``parcel'' method definition of LNB over estimates the level of maximum
detrainment by 3\,\unit{km} \citep{Takahashi:2012uq}. Section~\ref{apdx:lightning/ktop}
contains detailed discussions of the choice of 2\,\unit{km}
cloud-top reduction and how it compares to offline computations of
20\,\unit{dBZ} cloud tops. Alternative methods for estimating the difference
between the two heights can be formulated by directly taking into account
their respective definitions. However, echoing \cite{Barthe:2010uq}, such
addition of complexity increases the number of sources for uncertainty,
especially in the context of parameterized convection. Similarly, using
modeled cloud particle variables would also add an additional level of
sensitivity due to sub-grid variability in hydrometeor mixing ratios.
Therefore, reflectivity calculations are only performed in the 4\,\unit{km}
simulation and only for the purpose of redistributing lightning flashes
horizontally as described below.

For case 4 (Table~\ref{tab:lightning/setup}), convection is explicitly simulated with a
modified \citet{Lin:1983zr} microphysics scheme. Since no convective
parameterization is used, the resolved maximum vertical velocities
($w_{\max}$) within the convective core are utilized \citep{Barth:2012qf},
and Eq.~(\ref{eq:f_c(w)}) is used instead of Eq.~(\ref{eq:f_c(z)}) for
estimating flash rate. In addition, since a single storm may often cover
multiple model grids, flashes are redistributed to within regions with a
minimum reflectivity of 20\,\unit{dBZ} calculated using hydrometeor (rain,
snow, graupel) information that is now better constrained at 4\,\unit{km}.
The IC\,:\,CG ratio is prescribed using a coarse version of the
\cite{Boccippio:2001ys} 1995--1999 climatological mean, which was computed
using data from the Optical Transient Detector
\citep[OTD;][]{Christian:1996aa} and the National Lightning Detection Network
\citep[NLDN;][]{5173582}. Because PR92 developed Eq.~(\ref{eq:f_c(w)}) based
on data at 5\,\unit{km} resolution, no resolution scaling is done to this
simulation. Because this particular simulation was driven by the meteorology
of its own WRF outer domains, it is restarted ``cold'' on 2~August to be
consistent with the outer domain meteorology.

Most of the implementations used in these simulations are arguably
``untuned'' and not scaled to climatology or observations by any additional
tuning factors, with the exceptions of the 2\,\unit{km} cloud-top height
reduction used in the cases with parameterized convection and the prescribed
climatological IC\,:\,CG ratios in case 4. Therefore, the correctness and
predictiveness of the flash rate parameterization are not guaranteed at the
time of the simulation given the lack of supporting validations of PR92 at
the tested grid spacings. However, without feedback to the meteorology
(except in case 4) and providing sufficient linearity in the biases of flash
prediction, offline comparisons should reveal any tuning requirements for
operational and research uses.

\subsection{Data description}\label{ssec:lightning/data}

While desirable, event-by-event analysis would be technically challenging
because the simulation may not produce the same strength, timing, and
location of each convective event. Furthermore, an event-by-event analysis is
unnecessary in the context of a mesoscale upper tropospheric chemistry study,
of which the meaningful timescales often average biases from many individual
events. Therefore, a large area where thunderstorms commonly occur is
selected. The ``analysis domain,'' defined as 30\degr -- 45\degr N,
80\degr -- 105\degr W (Fig.~\ref{fig:lightning/domain}), is used for time series
and statistical comparisons.


\begin{figure*}[t]
\includegraphics[width=120mm]{gmd-2012-94-f02-new.pdf}
\caption{Spatial distribution of 2006 and 2011 JJA total precipitation in
millimeters. \textbf{(a)} and \textbf{(c)}~are NWS precipitation degraded to
12\,\unit{km} resolution. \textbf{(b)} and \textbf{(d)} are 36\,\unit{km}
WRF-simulated total precipitation over the same periods with data above water
surfaces masked out.} \label{fig:lightning/precipmap}
\end{figure*}



The predicted lightning properties depend strongly on how the model simulates
convection. Thus, in Sect.~\ref{ssec:lightning/precip}, WRF simulated precipitation
is compared against National Weather Service (NWS) precipitation products to
evaluate the model's skill in representing convective strengths. The data are
collected from radars and rain gauges and \mbox{improved} upon using a Multi-sensor
Precipitation Estimator (MPE). Manual post-analyses are then performed by
forecasters to identify systematic errors
(\url{http://www.srh.noaa.gov/abrfc/?n=pcpn\_methods}). The final data products
used here are mosaic CONUS precipitation maps from 12\,River Forecast Centers
(RFCs) during JJA 2006 and 2011. The data are gridded into 4\,\unit{km}
resolution and are available as 24\,h totals over a hydrological day
beginning and ending at 12:00~UTC.

The simulated CG flash counts, computed online as predicted total flashes
$\times$ predicted CG fraction, are compared against data from the Vaisala
US National Lightning Detection Network \citep[NLDN;][]{5173582}. The
network provides continuous multiyear CONUS and Canada coverage of
$>90\,{\%}$ of all CG flashes with ongoing network-wide upgrades
\citep{Orville:2002uq,Orville:2010uq}. The median location accuracy is
250\,\unit{m}, which is well within the resolutions employed in this study.
Multiple strokes are aggregated into a single flash if they are within 1\,\unit{s}
and no more than 10\,\unit{km} apart. Weak positive flashes with
$<15\,\unit{kA}$ have been filtered from all data. Finally, the flash data
are binned into hourly flash counts for each model grid cell for comparison
against model output.

Data from Earth Networks Total Lightning Network (ENTLN), previously
WeatherBug Total Lightning Network (WTLN), are used to validate the
model-produced IC\,:\,CG ratios. ENTLN employs a wide-band system that
operates between 1\,\unit{Hz} to 12\,\unit{MHz} \citep{Liu:2011vn}. The
theoretical detection efficiency (DE) for CG flashes across CONUS is
90--99\,{\%}, while the IC DE falls between 50--95\,{\%} (50--85\,{\%} within
the analysis domain). Since the mappings of the corresponding DEs are not
available with the data, 95\,{\%} and 65\,{\%} are used for CG and IC DEs,
respectively, in analyses for which a prescribed DE is required. To address
the concern of the impact of this simplification, the range of possible flash
counts, IC\,:\,CG ratios, and biases will be provided when appropriate within
the discussion in Sect.~\ref{sec:lightning/results} to place bounds on the
uncertainty. Due to the limited deployment duration of the network, only the
IC\,:\,CG ratios during JJA 2011 within the analysis domain (see
Fig.~\ref{fig:lightning/domain}) are estimated and compared. For consistency with the
comparisons against NLDN CG flash counts, the stroke aggregation criteria
used here are 10\,\unit{km} and 1\,s as done by NLDN, instead of the
10\,\unit{km} and 700\,\unit{ms} window typically used by Earth Networks to
generate flash statistics.

\begin{figure*}[t]
      \includegraphics[width=120mm]{gmd-2012-94-f03-new.pdf}
      \caption{Time series and frequency distributions for JJA 2006
          and 2011 area-averaged daily precipitation within the
          analysis region (see Fig.~\ref{fig:lightning/domain}). Distributions
          for NWS are scaled by the ratios between total grid counts in
          WRF at 36\,\unit{km} and total grid counts in NWS within the
          analysis boundaries ($\sim1/78$). WRF subgrid is the portion
          of precipitation from subgrid cumulus parameterization. Only
          grid points with more than 1 mm of precipitation are
          included.}
      \label{fig:lightning/precipseries}
\end{figure*}



\section{Results and discussions}\label{sec:lightning/results}

\subsection{Precipitation}\label{ssec:lightning/precip}

While lightning does not directly depend on precipitation, they are both the
results of the same processes that promote ice--graupel collisions. Further,
precipitation is observed robustly and continuously, thus giving us a high
quality measurement for validating model results. On the other hand, while
convective mass flux may produce a more consistent correlation with
lightning, the lack of well-controlled direct observations and the large
uncertainty in model calculations make it an inferior proxy for convective
strength in this context.

Figure~\ref{fig:lightning/precipmap} shows the total precipitation during JJA 2006 and
2011 over the CONUS as simulated at 36\,\unit{km} grid spacings by WRF and
observed by NWS. The gradients across the CONUS for both years are well
captured by the model, but WRF has a high bias for 2006. WRF also simulates
up to an order of magnitude more precipitation for coastal regions for both
years but primarily for 2006. The time series for mean daily area-averaged
precipitation and frequency distributions for JJA 2006 within the analysis
domain (Fig.~\ref{fig:lightning/precipseries}a and c) also reveal a median model bias
of 37\,{\%}. In particular, WRF predicted more than twice the precipitation
between late-June and mid-July in 2006. In contrast, the median bias for 2011
is 4.9\,{\%} with almost equal occurrence of over- and under-predictions. The
model frequency distribution for both years also closely track those observed
(Fig.~\ref{fig:lightning/precipseries}c and d) except at the high end of the
distribution where limits of model grid resolution induces significant noise.

\begin{figure*}[t]
      \includegraphics[width=120mm]{gmd-2012-94-f04.pdf}
      \caption{Total CG flashes in number per \unit{km^2} per
          full-year during JJA 2006 (first row) and 2011 (second
          row). First column (\textbf{a}~and \textbf{c}) shows the NLDN observed
          density gridded to WRF 36 model grid, and second column
          (\textbf{b} and \textbf{d}) shows the modeled flash density output by
          WRF at 36\,\unit{km}.}
      \label{fig:lightning/cgmap}
\end{figure*}


The simulated daily precipitation at 12\,\unit{km} is higher than the NWS
observed precipitation by 24\,{\%} during July 2011. However, an anomalously
strong diurnal cycle is simulated at 12\,\unit{km} grid spacing that is not
present in the 36\,\unit{km} simulation. Comparing the area-averaged
12\,\unit{km} nocturnal precipitation over the entire analysis domain to that
of 36\,\unit{km} output, nocturnal precipitation at 12\,\unit{km} is too low
but the daytime precipitation is too high. One-day simulations were
performed to evaluate the impact from the differences in model physics, but
there remained significant unidentified discrepancies between the
precipitation amount in the two runs that cannot be explained by horizontal
resolution differences alone; thus, it is concluded that there is no value in
redoing the entire simulation. The identified causes for the differences
between the two simulations are, in decreasing order for magnitude of
influence, initial conditions for soil temperature and soil moisture,
differences in planetary boundary layer scheme (Sect.~\ref{ssec:lightning/model}), and
the land surface model option. Such difference in diurnal behavior in the
simulations is expected to have significant impact on how the lightning
parameterization is evaluated, but the full impact can be minimized through
incorporation of precipitation into the analysis.

Large scale meteorology and moisture inputs are unlikely the causes for the
these biases due to nudging. In 2006, biases mostly occurred in the
low-to-mid end of the distribution (i.e., light precipitation events,
Fig.~\ref{fig:lightning/precipseries}c), indicating that the problem lies in
parameterizing subgrid events. Despite the differences in the eastern United
States, convection over the central United States is reasonably represented.
Finally, the goal of this study is not to evaluate the convective
parameterization nor specific model setup, but rather to evaluate the
performance characteristics of the lightning parameterization when
implemented into a regional model with all the expected (and unexpected)
defects.


\begin{figure*}[t]
\includegraphics[width=120mm]{gmd-2012-94-f05.pdf}
\caption{Comparisons of time series and frequency distributions between NLDN
CG flash counts (black) and WRF predicted CG flash counts (red) at
36\,\unit{km} within the analysis domain defined in Fig.~\ref{fig:lightning/domain}.
Total flash counts predicted by WRF are shown as dotted red lines.}
      \label{fig:lightning/cgseries}
\end{figure*}

\begin{figure}[t]
\includegraphics[width=80mm]{gmd-2012-94-f06.pdf}
\caption{Total CG flashes (\#) versus area-mean daily precipitation
(\unit{mm}) within the analysis domain (Fig.~\ref{fig:lightning/domain}). Solid line is
the least-square linear fit and dashed lines are $\pm1\sigma$ for both the
constant terms and first-order coefficients. WRF is simualted at
36\,\unit{km}.}
      \label{fig:lightning/cgslope}
\end{figure}

\subsection{CG flash rate}\label{ssec:lightning/flashrate}

Figure~\ref{fig:lightning/cgmap} shows the CONUS CG flash density (units in number per
\unit{km^2} per year). WRF is consistently higher along the East Coast for
2006 where positive bias is also observed in the modeled precipitation, which
is used as a proxy for quantifying the comparison of simulated convective
strength against observations. Similarly, both flash rate and precipitation
are over-predicted in the Colorado and New Mexico region for 2006. On the
other hand, the low precipitation bias in Arizona simulated by WRF for 2011
is coincident with a severe low bias in the same region for the CG flash
density. Otherwise, flash densities are within the order of magnitude of
those observed for regions where simulated precipitation is consistent with
NWS observations.

The over-prediction of CG flash density along the East Coast in 2006
dominates the regional mean and produces significantly high biases compared
to 2011. Figure~\ref{fig:lightning/cgseries}a and \ref{fig:lightning/cgseries}b show the time
series of the total number of ground flashes predicted by WRF and observed by
NLDN within the analysis region (Fig.~\ref{fig:lightning/domain}). The median daily CG
bias is 140\,{\%} for 2006 and only 13\,{\%} for 2011. It should be noted
that these values were obtained by sampling all 3\,months. Sampling one month
would produce varying results. For instance, while JJA 2011 produces an
overall median bias of 13\,{\%}, July 2011 alone produces about twice as much
lightning as observed but is offset by under-predictions in June and August.
Because the lightning detection efficiency of NLDN varies spatially, the CG
bias can vary over ranges of 116--154\,{\%} for 2006 and 1.7--20\,{\%} for 2011.
The differences between the median biases for the two
summers can be attributed largely to the differences in the total
precipitation biases, as illustrated in the previous section
(Sect.~\ref{ssec:lightning/precip}), for which 2006 is 37\,{\%} too high while 2011 is
only 5\,{\%} higher than observations.

To take into account the bias in the simulated convective strength,
area-averaged daily precipitation is correlated with total CG flash count.
While the relation is likely nonlinear, the area averages over the analysis
domain are roughly linear in both WRF-simulated and observed data
(Fig.~\ref{fig:lightning/cgslope}). The slopes for the 2006 data are statistically the
same, there is a constant positive bias for model produced flash counts over
observed values. In contrast, 2011 results are close for small values but
modeled and observed values diverge for more intense events. Such
inconsistency between years demonstrates the potential for strong
inter-annual variability in the correlation between flash rate and
precipitation.



\begin{figure}[t]
\includegraphics[width=80mm]{gmd-2012-94-f07.pdf}
\caption{Total lightning and CG flash rates computed using PR92 and PR93 for
various cloud-top heights and freezing levels, demonstrating the source of
spectral cut-off in Fig.~\ref{fig:lightning/cgslope}.} \label{fig:lightning/pr9293}
\end{figure}

\begin{figure}[t]
\includegraphics[width=80mm]{gmd-2012-94-f08.pdf}
\caption{IC\,:\,CG bulk ratios for JJA 2011 as \textbf{(a)} observed by
 ENTLN and \textbf{(b)} predicted by WRF at 36\,\unit{km} grid spacing
 using PR93. The ENTLN detection efficiency used here is
 0.65 for IC and 0.95 for CG. }
 \label{fig:lightning/iccgmap}
\end{figure}






Figure~\ref{fig:lightning/cgseries}c and d show the frequency
distributions of the hourly grid flash density. From the spectra, it is
apparent that the over-prediction observed in the time series occurs between
flash densities of 0.003 to 0.1\,\unit{CG\,flashes\,km^{-2}\,h^{-1}}.
However, the abrupt cutoff beyond $\sim0.11$ in both 2006 and 2011 modeled
distribution indicates that PR92 fails to replicate the observed
distribution. The occurrence of this cutoff can be explained by the local
maximum when combining the PR92 total flash rate parameterization and PR93
IC\,:\,CG ratio parameterization (Fig.~\ref{fig:lightning/pr9293}). Together, the
predicted CG flash rate is capped at a certain limit, depending on the
freezing level regardless of the cloud-top height. In addition, the total
flash rate is also under-predicted for high flash rate events (dotted red
lines in the figures), thus contributing to the truncated model frequency
distribution.

An initial comparison of the model results against the Lightning Imaging
Sensor (LIS) total flash data also shows a high bias ($\sim2.3\times$) in
overall flash count but an underestimation of the high flash count events,
similar to the results demonstrated by the NLDN comparison. However,
comparisons against the LIS data for this study have a low confidence level
because of the relatively short time period simulated and the many
uncertainties, such as variable detection efficiency and shifting diurnal
sampling bias of LIS data, associated with the analysis.

\subsection{IC\,:\,CG ratio}\label{ssec:lightning/ratio}

The JJA 2011 IC\,:\,CG bulk ratios (\,$\equiv\sum_t \textrm{IC}(\vec{x},t)/\sum_t
\textrm{CG}(\vec{x},t)$\,) are calculated within the analysis domain
(Fig.~\ref{fig:lightning/iccgmap}a) using constant detection efficiencies of 95\,{\%}
and 65\,{\%} for CG and IC flashes, respectively. While WRF produced a median
IC\,:\,CG ratio of 1.74 within the region, ENTLN observed a median of 5.24
with a possible range of 3.80 to 7.17 due to the spatial variability in both
IC and CG DEs. Considering the ambiguity in the choice of cloud-top
definition described in Sect.~\ref{ssec:lightning/model}, a possible solution to
increase the IC\,:\,CG ratio computed using Eq.~(\ref{eq:Z}), thus achieving
better comparison against observations, is by eliminating the cloud-top
height reduction, an option that maintains the conceptual interpretation of
the parameterization but has the potential of offsetting the bias. For
consistency, the cloud-top height used in the total lightning
parameterization needs to be un-adjusted as well.

To learn whether reasonable lightning flash rates and IC\,:\,CG ratios can be
estimated by using just the level of neutral buoyancy (LNB), an offline
calculation is made of the daily flash counts with the cloud-top height
adjustment eliminated. The offline calculation is performed using
instantaneous, hourly model output of LNBs and temperatures (for determining
freezing levels). While the offline calculation is able to replicate almost
precisely the online flash count prediction, which causes both time series to
appear overlapping in Fig.~\ref{fig:lightning/offcompare}, the CG flash rate
frequency distribution is severely degraded because of vertical
discretization of cloud tops to model levels and lowered temporal resolution
to hourly outputs. When LNB is used for the cloud-top height (with no
adjustment), the prediction of both CG and total lightning flash rates
increase, as expected. The CG median bias over ENTLN increases from 44 --
51\,{\%} to 158 -- 172\,{\%}, and the total lightning median negative bias of
53 -- 25\,{\%} becomes a positive bias of 23 -- 95\,{\%} for the
aforementioned range of DEs. Furthermore, even though the frequency
distribution of total lightning is closer to the observed distribution, the
CG distribution still experiences the truncation as described in
Sect.~\ref{ssec:lightning/flashrate}.

\begin{figure*}[t]
\includegraphics[width=140mm]{gmd-2012-94-f09.pdf}
\caption{Comparison of WRF predicted lightning flash counts generated online
and offline with and without $-2$\,\unit{km} cloud-top height adjustments
against ENTLN CG and total flash counts. Thicknesses of the ENTLN bands in
the time series are computed using the minimum and maximum theoretical IC and
CG detection efficiencies within the analysis domain. Noisiness of offline
calculated distributions are associated with using hourly outputs only rather
than accumulating flashes at every model time step. It should be noted that
online (black) and offline (blue) WRF outputs with adjusted top appear
coincident in the time series, but are evidently different from the frequency
distribution.}
      \label{fig:lightning/offcompare}
\end{figure*}

\begin{figure}[t]
      \includegraphics[width=80mm]{gmd-2012-94-f10.pdf}
      \caption{Time series and frequency distributions of 3-hourly CG flash counts compared to NLDN at gridded to 12\,\unit{km}.
      The WRF 36\,\unit{km} distribution is adjusted by $\times9$ to account for the grid per area difference. The choice of
      computing the distributions for flash rate per grid as opposed to flash density is to demonstrate the consistency of
      the spectral drop-off at different resolutions.}
      \label{fig:lightning/gabiseries}
\end{figure}

\begin{figure}[t]
      \includegraphics[width=80mm]{gmd-2012-94-f11.pdf}
      \caption{Time series and frequency distributions of hourly CG flash counts within
      the analysis domain as observed by NLDN and simulated by WRF at 4\,\unit{km} grid spacing.}
      \label{fig:lightning/4kmseries}
\end{figure}

\section{Resolution dependency}\label{sec:lightning/resolution}

A goal of this study is to evaluate the applicability of the PR92
parameterization to resolutions between fully parameterized and partially
resolved convection. Thus, it is useful to evaluate how the parameterization
behaves as the grid size changes. To test the behavior of the PR94
calibration factor, a 12\,\unit{km} simulation for July 2011 is used. As grid
sizes are reduced to allow convective parameterization to be turned off, the
transition to $w_{\max}$ based formulation of PR92 (Eq.~\ref{eq:f_c(w)}) is
tested with a 4\,\unit{km} simulation between 25~July--7~August~2006. The
domains for these simulations are shown in Fig.~\ref{fig:lightning/domain}. Together,
the results from these simulations will provide insights and recommendations
on how to achieve resolution-awareness or independence while using PR92.

\subsection{Sensitivity to grid size}\label{ssec:lightning/gridsens}

At 12\,\unit{km}, the resolution dependency factor or ``calibration factor''
($c$) from \cite{Price:1994fk} is 0.56\,{\%} smaller than that applied to
36\,\unit{km}. However, comparison against the 36\,\unit{km} simulation and
observations shows that there is a factor of $\sim10$ high bias. While there
are differences in the statistics of convective strengths between the two
simulations, as quantified by precipitation in Sect.~\ref{ssec:lightning/precip},
they are too minor to fully reconcile the large bias at 12\,\unit{km}.
Therefore, an areal ratio scaling factor ($1/9 = 12^2/36^2$) is applied
offline to partially reconcile the differences on top of $c$($\sim1$), which
was applied online.

There are two reasons why the use of areal scaling instead of PR94 is
justified in this study. The first reason pertains to why PR94 failed while
it has been shown to work in GCMs. The PR94 calibration factor was derived
from area-averaged cloud-top heights for progressively larger grid sizes from
the original ISCCP 5\,\unit{km} resolution to $8^\circ\times10^\circ$. On the
contrary, the LNBs from the convective parameterization are expected to
change only slightly with grid resolution as long as the environmental
parameters such as buoyancy remain similar.

The second reason addresses why the areal ratio is expected to work for
regional scales. PR92 produces flash counts in unit of number of flashes per
storm; thus, when approaching almost convection-resolving resolutions, where
major storm size is comparable to grid size, the appropriate scaling should
be done according to the expected number of convective cores per grid. Since
$\Delta x=36$\,\unit{km} gives a reasonable flash rate compared to
observations over 3\,months, we assumed one storm per grid at this resolution
and scaled the flash counts from $\Delta x=12$\,\unit{km} as an areal ratio.
However, the base case resolution may spatially vary because of the dynamics
controlling the minimum-permitted distance between convective cores. At
coarse grid sizes, the area covered by the convective storm systems may only
be a fraction of the grid cell area. Thus, the area scaling ratio may not be
applicable when changing from base-case grid spacing (with approximately one
storm per grid) to much coarser grid sizes. A possible solution is to include
a cloud fraction estimate as part of the scaling factor between grid sizes.

After scaling by $1/9$, WRF at 12\,\unit{km} predicts a median of 40\,{\%}
more 3-hourly lightning flashes than observed by NLDN
(Fig.~\ref{fig:lightning/gabiseries}). This is to be compared with 36\,\unit{km}, which
predicted double the 3-hourly lightning for the same period. Simulating an
anomalously strong diurnal cycle in precipitation, the 12\,\unit{km} flash
count also shows a much more prevalent diurnal variation, associated with the
poor simulation of the diurnal cycle of precipitation as previously noted.
Much of the over-prediction is compensated by the negative biases in the
nocturnal flash rates in the final statistics. Despite the differences in
diurnal skill, the parameterization was able to produce the same drop-off in
grid frequency distribution beyond 200 flashes per grid per 3\,h, for which
the primary cause is discussed in Sect.~\ref{ssec:lightning/flashrate}.

\subsection{Sensitivity to formulation}\label{ssec:lightning/formsens}

Comparing the 36\,\unit{km} simulation to the 4\,\unit{km} simulation
provides insight into how the predicted flash density changes between
resolutions using $f(z_{\mathrm{top}})$ for parameterized convection and
$f(w_{\max})$ for resolved convective systems. This is an important factor to
be considered if flash rate predictions are to be included in nested
simulations or models permitting non-uniform grid-spacings such as Model for
Prediction Across Scales-Atmosphere \citep[MPAS-A;][]{Skamarock:2012fk}.

The area-averaged daily precipitation predicted by the 4\,\unit{km} WRF-Chem
simulation is 70\,{\%} too high prior to 2~August~2006 and only 7.5\,{\%} too high
after 2~August. On 2~August, the 4\,\unit{km} WRF simulation was
re-initialized (with no clouds) to be consistent with the re-initializations
of the outer domain WRF simulations that drove this 4\,\unit{km} simulation
described in \cite{Barth:2012qf}. The flash rate predicted by the
4\,\unit{km} simulation follows the precipitation trend. A 26\,{\%} decrease
in flash rate occurs between the period before 2~August and the period
afterwards.

While the 36\,\unit{km} simulation over-predicted lightning flash rate for
this period~(25~July--7~August~2006), the 4\,\unit{km} simulation
under-predicted the flash rate, exhibiting a $-83\,{\%}$ bias relative to the
NLDN flash counts prior to the cold-start and a $-95\,{\%}$ bias after
(Fig.~\ref{fig:lightning/4kmseries}). Similar underestimation of the $w_{\max}$
formulation has been noted for both tropical (Hector storm near Darwin,
Australia) and US continental storms \citep{Cummings:2012ly}. These results
indicate that it is important to evaluate the flash rate parameterizations
with observations. It is insufficient to use high resolution model results as
``truth'' for coarse resolution simulations.

Despite the low bias in flash rate prediction, the 4\,\unit{km} WRF-Chem
simulation matches the observed distribution of flashes for high flash rate
events and placed the burden of underestimation on the low-end of the
distribution, which causes the distribution to appear flatter than observed.
Since we are using a constant IC\,:\,CG ratio based on
\citet{Boccippio:2001ys} climatology instead of the PR93 parameterization,
the erroneous drop-off in the CG flash rate distribution found in the other
cases using PR93 is not present. Such improvement in spectral characteristics
suggests that constant climatological IC\,:\,CG ratios may be a reasonable if
not superior alternative to PR93.

\conclusions\label{sec:lightning/conclusions}

We have implemented the WRF-Chem model parameterizations for lightning
flash rate using prescribed IC\,:\,CG ratios and the associated resolution dependency by
\citet{Price:1992wb,Price:1993fk,Price:1994fk}, which are based on cloud-top
height. In our implementation, the cloud-top height is estimated by the level
of neutral buoyancy (LNB), adjusted by $-2\,\unit{km}$ to reconcile the
difference between LNB and radar reflectivity cloud top. No additional
tunings and changes to the parameterizations are done. The modeled
precipitation and \mbox{lightning} flash rate are evaluated for the simulations with
36\,\unit{km}, 12\,\unit{km}, and 4\,\unit{km} grid spacings over CONUS for
JJA 2006 and 2011.

The first result is that, after a 2\,\unit{km} reduction, the use of LNB as a
proxy for cloud-top simulated at 36\,\unit{km} grid spacing produces CG flash
rates at the same order of magnitude as NLDN observations. For models using
other convective parameterizations, alternative choices of cloud-top proxies
may be available and thus the appropriate methods of cloud-top adjustment
should be determined on a case-by-case basis. Taking into account model
biases in convection, as quantified by precipitation, the
precipitation--lightning relation from the model and observations are
statistically indistinguishable. While there is up to a factor of $2.4$
median bias in the flash counts from the 2006 36\,\unit{km} simulation, it is
accompanied by a 37\,{\%} over-prediction in precipitation. In contrast, the
2011 36\,\unit{km} simulation has a precipitation bias of 5\,{\%}, which
leads to a 13\,{\%} over-prediction in flash counts. For the 12\,\unit{km}
simulation the lightning flash rate bias is linked to the anomalously strong
diurnal cycle simulated for convection, indicated by precipitation. Such bias
in the simulated convection may be caused by a number of other model
components.

Second, despite the correct CG count, it is shown that PR92 is incapable of
producing the correct frequency distribution flash CG flashes, which are
truncated at a much lower flash density than observed. The most likely cause
is the function form of combining the PR92 total flash rate parameterization
and the PR93 IC\,:\,CG ratio parameterization, which produces an upper-limit
in the permitted maximum CG flash rate. This brings into question the
validity of PR93 in contexts where spectra characteristics are a concern. It
is recommended that using constant bulk ratios such as the climatology
presented in \citet{Boccippio:2001ys} or one derived from total lightning
measurements may produce equal, if not better, spectra. Considering that the
observed JJA 2011 IC\,:\,CG ratio also displays significant departure from
the \citet{Boccippio:2001ys} climatology for certain areas, it would be
useful to revisit the subject of IC\,:\,CG climatology in future studies,
taking
advantage of the advances in continuous wide-area lightning detections over
the past two decades.

Third, due to the use of LNB from the convective parameterization instead of
area-averaged cloud-top heights, using the PR94 factor to adjust for different
horizontal resolutions is not applicable and an areal ratio factor should be
used instead to reconcile the resolution dependencies. Since the 36\,\unit{km} base
cases produced relatively satisfying results, the 12\,\unit{km} simulation is
scaled with an areal ratio relative to 36\,\unit{km}. However, it may be
argued that the outcome from the 36\,\unit{km} simulations is only a
corollary of the probability of having exactly one convective core within a
single model grid. Therefore, other choices of ``base case'' grid spacings
near 36\,\unit{km} may also produce similar results for CONUS, specifically
within the analysis domain (Fig.~\ref{fig:lightning/domain}), and other areas with
different storm density may require a different base-case resolution for
scaling. On the other hand, area ratios may not be appropriate at coarser
resolution as convective core number density is highly non-uniform.

Finally, at 4\,\unit{km}, we used a theoretically similar formulation of PR92
based on $w_{\max}$ within convective cores identified as regions with
20\,\unit{dBZ} or greater radar reflectivity. While the parameterization
includes the high flash rate storms, thereby giving a frequency distribution
shaped similar to that observed without the erroneous drop-off, the flash
count is under-predicted by up to a factor of 10. From this experiment, we
see the need to evaluate flash rate parameterizations with observations for
the locations and periods specific to the simulations. It is insufficient to
use high resolution model results as ``truth'' for coarse resolution
simulations. Hence, validation and tuning prior to further usage of
$f(w_{\max})$ from Eq.~(\ref{eq:f_c(w)}) is encouraged. Furthermore,
parameterizing flash rate in cloud-resolving models based on other storm
parameters \citep{Barthe:2010uq} should also be tested.

To summarize, we recommend the following when applying the Prince and Rind
parameterizations for lightning flash rates:
\begin{enumerate}
\item Proper adjustment to cloud top should be made to match the expected 20~\unit{dBZ} radar reflectivity top when applying PR92;
 \item PR93 for IC\,:\,CG partitioning should be used only if it is unnecessary to get information on the frequency distribution of flashes;
  \item Scaling for resolution dependency may be performed by areal ratio against a base-case resolution, defined as that producing 1 storm per grid within the domain of interest (36\,\unit{km} grid spacing in this study).
  \end{enumerate}

To further the confidence of the lightning flash rate parameterizations and
IC\,:\,CG partitioning, long-term wide-area total lightning detection and
data archiving should be accompanied by coincident observations of cloud-top
or other convective properties with well-defined error characteristics in
observations and quantifiable predictability in numerical models.

\section{Comments on cloud-top height reduction} \label{apdx:lightning/ktop}

In this study, we used the level of neutral buoyancy (LNB) from the WRF
implementation of the Grell--Devenyi convective parameterization
\citep{Grell:2002bs} as a proxy for sub-grid cloud-top heights for the
purpose of testing a flash rate parameterization by
\citet{Price:1992wb,Price:1993fk,Price:1994fk}. A reduction of 2\,\unit{km}
is used to reconcile the differences between LNB and the cloud top that would
be obtained if defined at a 20\,\unit{dBZ} reflectivity threshold.
While this method produces an integrated flash count consistent with that
observed after taking into account model biases in convective precipitation,
we acknowledge that storm-to-storm variability cannot be captured by such a
simple \mbox{approach}. Presented in this section are offline calculations of both
20\,\unit{dBZ} cloud tops and LNB cloud tops from a 13-day simulation at
4\,\unit{km} grid spacing to understand the margin of potential errors.

Radar reflectivity is estimated by using rain, snow, and graupel particle
information from hourly outputs. For consistency, the offline calculation of
reflectivity uses the same modified equations from \citet{Smith:1975fk} and
criteria as those used in the 4\,\unit{km} simulation. The highest model
level with more than 20\,\unit{dBZ} is then defined as the 20\,\unit{dBZ}
top.

LNB is estimated by a simple ``parcel method,'' rather than emulating the
full algorithm in the parameterization as implemented in WRF. Therefore, the
result may differ from what would be produced within the model. First, the
dew point depression at the surface model level is determined, which is then
used to seek the lifting condensation level (LCL) assuming adiabatic ascent.
From the LCL, the moist adiabatic lapse rate ($\Gamma_\textrm{m}$) is calculated and
the level of free convection (LFC) is determined by linearly extrapolating
the moist adiabat using the lower level's $\Gamma_\textrm{m}$ to the model level
immediately above. From the LFC, a search is performed at incremental model
levels until the LNB is exceeded. Grid points with LFCs $< 500\,\unit{m}$ or
above-freezing temperature at LNBs are discarded.

In total, $1.34\times10^6$ columns with sufficient reflectivity and cloud-top
heights greater than 5\,\unit{km} AGL are found. The distribution of the
difference $\langle h_\textrm{LNB}-h_\textrm{dBZ}\rangle$ indicates that LNB is higher than
the 20\,\unit{dBZ} top 62\,{\%} of the time with a mean of $1.1$\,\unit{km}
and a standard deviation of $2.3$\,\unit{km}. Other metrics for defining the
required offsets between the two heights can produce different results. For
example, to minimize the bias after applying PR92 with
$\left|\langle(h_\textrm{LNB}-\delta h)^5/h_\textrm{dBZ}^5\rangle-1\right|$, the reduction
$\delta h$ evaluates to 3.27\,\unit{km}. While the 2\,\unit{km} reduction
used in this study differs from the two computed here, it is within the
calculated range and thus can still be considered a median representation,
especially when the uncertainties in the methods used for the offline LNB and
radar reflectivity computations are taken into account.

Finally, it is essential to re-emphasize that the choice of cloud-top
reduction is specific to the use of the Grell--Devenyi convective
parameterization in WRF or other models producing LNB as the best-available
proxy for sub-grid cloud-tops. In other models, cloud-top proxies other than
LNB may be present. In those cases, an adjustment specific to those proxies
should be used if PR92 is the preferred method for parameterization. An
alternative would be to reformulate PR92 to be based on LNB, which lies
beyond the scope of this paper and may be attempted in the future as needed.